\documentclass[12pt, a4paper]{article}
\usepackage[T2A]{fontenc}
\usepackage[utf8]{inputenc}
\usepackage[russian]{babel}
\usepackage{amsmath}
\usepackage{amssymb}
\usepackage{graphicx}
\usepackage{setspace}
\onehalfspacing
\setlength\parindent{0pt}
\usepackage[top=2cm,bottom=2cm,left=1.5cm,right=1.5cm]{geometry}

\begin{document}

\subsubsection*{Влажный воздух}

\[
p_\text{sat}(T = 100^\circ C) = p_0 \approx 100\; \text{кПа}.
\]

\subsubsection*{Изотерма \(p(V)\) для водяного пара}

\[
p(V)=\frac{\nu R T}{V}, \quad p < p_{\text{sat}}(T).
\]

\[
p_{\text{sat}}(T)=\frac{\nu R T}{V_{\text{sat}}} \quad \Longrightarrow \quad V_{\text{sat}}=\frac{\nu R T}{p_{\text{sat}}(T)}.
\]

\[
p(V)=p_{\text{sat}}(T)=\text{const}, \quad V \in [V_{\ell}, V_{\text{sat}}].
\]

\subsubsection*{Изотерма \(p(V)\) для влажного воздуха}

\[
p(V)=p_d(V)+p_v(V).
\]

\[
p(V)=\frac{(\nu_d+\nu_v) R T}{V}, \quad p_v < p_{\text{sat}}(T).
\]

\[
p_v = p_{\text{sat}}(T) \quad \Longrightarrow \quad V_{\text{sat}} = \frac{\nu_v R T}{p_{\text{sat}}(T)}.
\]

\[
p(V) = \frac{\nu_d R T}{V} + p_{\text{sat}}(T).
\]

\[
\frac{dp}{dV} =
\begin{cases}
-\dfrac{(\nu_d+\nu_v) R T}{V^2}, & V > V_{\text{sat}} \\
-\dfrac{\nu_d R T}{V^2}, & V < V_{\text{sat}}
\end{cases}
\]

\end{document}