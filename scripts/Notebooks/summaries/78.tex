\documentclass[12pt, a4paper]{article}
\usepackage[T2A]{fontenc}
\usepackage[utf8]{inputenc}
\usepackage[russian]{babel}
\usepackage{amsmath}
\usepackage{graphicx}
\usepackage{setspace}
\onehalfspacing
\setlength\parindent{0pt}
\usepackage[top=2cm,bottom=2cm,left=1.5cm,right=1.5cm]{geometry}

\begin{document}
\subsubsection*{Принцип Ферма и показатель преломления}
\[
\mathcal S=\int n(\vec r)\,ds, \quad \delta\mathcal S=0, \quad n=\frac{c}{v}.
\]

\subsubsection*{Закон Снеллиуса (Закон преломления света)}
\[
n_1\sin\theta_1 = n_2\sin\theta_2.
\]

\subsubsection*{Базовые значения показателя преломления}
\[
n_{\text{воздух}}\approx1.0003,\quad
n_{\text{вода}}\approx1.33,\quad
n_{\text{стекло}}\approx1.50.
\]
\end{document}