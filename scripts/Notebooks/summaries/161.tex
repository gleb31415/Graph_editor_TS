\documentclass[12pt, a4paper]{article}
\usepackage[T2A]{fontenc}
\usepackage[utf8]{inputenc}
\usepackage[russian]{babel}
\usepackage{amsmath, amsfonts, amssymb}
\usepackage{graphicx}
\usepackage{wrapfig}
\usepackage{setspace}
\onehalfspacing
\setlength\parindent{0pt}
\usepackage[top=2cm, bottom=2cm, left=1.5cm, right=1.5cm]{geometry}

\begin{document}

\subsubsection*{Ввод энтропии для квазистатического процесса}

\[
dS = \frac{\delta Q_{\text{кв.ст.}}}{T}, \quad S_1 - S_0 = \int\limits_{0 \rightarrow 1} \frac{\delta Q}{T}
\]

\subsubsection*{Ограничения для энтропии}

\[
\int\limits_{1\rightarrow2}^{\mathrm{I}} \frac{\delta Q}{T} \le \int\limits_{1\rightarrow2}^{\mathrm{II}} \frac{\delta Q_{\text{кв.ст.}}}{T} = S_2 - S_1, \quad S_2 \ge S_1
\]

\subsubsection*{Энтропия идеального газа}

\[
\Delta S = \nu C_V \ln\left(\frac{T}{T_0}\right) + \nu R \ln\left(\frac{V}{V_0}\right)
\]

\subsubsection*{Извлечение максимальной работы}

\[
A = C \left(\sqrt{T_{20}} - \sqrt{T_{10}}\right)^2, \quad T_1 T_2 = T_{10} T_{20}
\]

\subsubsection*{Координаты $TS$}

\[
Q = \int \delta Q = \int T dS
\]

Для цикла Карно:

\[
\eta = \frac{A}{Q_+} = \frac{T_2 - T_1}{T_2}
\]

\textbf{Пример 1. КПД цикла}

\[
\eta = \frac{C_P (T_3 - T_2) - C_V (T_3 - T_1)}{C_P (T_3 - T_2)}
\]

\textbf{Пример 2. Фазовый переход на $TS$ диаграмме}

\begin{itemize}
\item $1 \rightarrow 2$ – фазовый переход жидкость–газ
\item $2 \rightarrow 3$ – изотерма (газ)
\item $3 \rightarrow 4$ – изобара (газ)
\item $4 \rightarrow 5$ – конденсация
\item $5 \rightarrow 6$ – изохора (жидкость)
\item $6 \rightarrow 1$ – изотерма (жидкость)
\end{itemize}

\end{document}