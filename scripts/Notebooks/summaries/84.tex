<latex>
\documentclass[12pt, a4paper]{article}
\usepackage[T2A]{fontenc}
\usepackage[utf8]{inputenc}
\usepackage[russian]{babel}
\usepackage{amsmath}
\usepackage{amssymb}
\usepackage{graphicx}
\usepackage{setspace}
\onehalfspacing
\setlength\parindent{0pt}
\usepackage[top=2cm,bottom=2cm,left=1.5cm,right=1.5cm]{geometry}

\begin{document}

\subsubsection*{Формулировки ВНТ}

\begin{itemize}
  \item \textit{Формулировка Кельвина:} Невозможен циклический двигатель, полностью превращающий теплоту $Q_H$ одного источника в работу.
  \item \textit{Формулировка Клаузиуса:} Невозможен процесс, единственным результатом которого является передача теплоты от холодного тела к горячему.
\end{itemize}

\subsubsection*{Клаузиус $\Rightarrow$ Кельвин}

\[
K:\quad Q_H \to W=Q_H, \quad \text{без других эффектов.}
\]
\[
R:\quad W_R + Q_C \to Q_H^{(R)} = Q_C + W_R.
\]
При $W_R = W = Q_H$:
\[
\text{на } H: -Q_H + Q_H^{(R)} = 0, \quad \text{на } C: -Q_C, \quad \text{работа внешняя }=0.
\]

\subsubsection*{Кельвин $\Rightarrow$ Клаузиус}

\[
R:\quad Q \to Q, \quad W=0,
\]
\[
K:\quad Q_H \to W_K, \quad Q_C, \quad Q_H = W_K + Q_C.
\]
При $Q_C = Q$:
\[
\text{на } C: +Q - Q_C = 0, \quad \text{на } H: -Q_H + Q_C = -W_K,
\]
внешняя работа: $W_K$.

\end{document}
</latex>