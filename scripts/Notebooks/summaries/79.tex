\documentclass[12pt, a4paper]{article}
\usepackage[T2A]{fontenc}
\usepackage[utf8]{inputenc}
\usepackage[russian]{babel}
\usepackage{amsmath}
\usepackage{amssymb}
\usepackage{graphicx}
\usepackage{setspace}
\onehalfspacing
\setlength\parindent{0pt}
\usepackage[top=2cm,bottom=2cm,left=1.5cm,right=1.5cm]{geometry}

\begin{document}

\subsubsection*{Тонкие линзы и их свойства}

\textit{Главная оптическая ось (ГОО)} — прямая через центры кривизны. \textit{Оптический центр $O$} — точка пересечения линзы с ГОО.

\subsubsection*{Формула для фокуса тонкой линзы}

Пусть линза с показателем преломления $n>1$ и радиусами кривизны $r_1$, $r_2$. Для параксиального луча:

\[
\frac{1}{F} = (n-1)\left(\frac{1}{r_1} + \frac{1}{r_2}\right).
\]

В стандартной оптике с направленными радиусами:

\[
\frac{1}{F} = (n-1)\left(\frac{1}{R_1} - \frac{1}{R_2}\right).
\]

\end{document}