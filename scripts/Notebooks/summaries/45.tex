\documentclass[12pt, a4paper]{article}
\usepackage[T2A]{fontenc}
\usepackage[utf8]{inputenc}
\usepackage[russian]{babel}
\usepackage{amsmath, amsfonts, amssymb}
\usepackage{setspace}
\usepackage{graphicx}
\usepackage{geometry}
\onehalfspacing
\setlength\parindent{0pt}
\geometry{top=2cm,bottom=2cm,left=1.5cm,right=1.5cm}

\begin{document}

\subsubsection*{Столкновения}

\[
\sum \vec p_i = \text{const}
\]

\subsubsection*{Абсолютно упругий удар}

\[
m_1\vec v_{1}+m_2\vec v_{2}=m_1\vec v_{1}'+m_2\vec v_{2}',
\quad
\frac12m_1v_{1}^2+\frac12m_2v_{2}^2
=\frac12m_1{v_{1}'}^2+\frac12m_2{v_{2}'}^2.
\]

Для прямолинейного удара:

\[
v_{1}'=\frac{m_1-m_2}{m_1+m_2}v_{1}+\frac{2m_2}{m_1+m_2}v_{2}, 
\quad
v_{2}'=\frac{2m_1}{m_1+m_2}v_{1}+\frac{m_2-m_1}{m_1+m_2}v_{2}.
\]

\subsubsection*{Абсолютно неупругий удар}

\[
\vec v_c=\frac{m_1\vec v_{1}+m_2\vec v_{2}}{m_1+m_2}.
\]

\subsubsection*{Система отсчёта центра масс}

\[
\vec p^{(C)} = \sum_i m_i(\vec v_i - \vec v_C) = 0.
\]

\end{document}