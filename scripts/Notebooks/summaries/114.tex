\documentclass[12pt, a4paper]{article}
\usepackage[T2A]{fontenc}
\usepackage[utf8]{inputenc}
\usepackage[russian]{babel}
\usepackage{amsmath}
\usepackage{amssymb}
\usepackage{graphicx}
\usepackage{setspace}
\onehalfspacing
\setlength\parindent{0pt}
\usepackage[top=2cm,bottom=2cm,left=1.5cm,right=1.5cm]{geometry}

\begin{document}

\subsubsection*{Определение равновесия}
\[
F(x_0)=0, \quad F(x)=-\frac{dU}{dx}, \quad \left.\frac{dU}{dx}\right|_{x_0}=0.
\]

\subsubsection*{Энергетический критерий устойчивости}
\[
U(x_0+\xi) \approx U(x_0) + \frac12 U''(x_0) \xi^2,
\quad
\text{устойчиво} \Leftrightarrow U''(x_0) > 0.
\]

\subsubsection*{Силовой критерий устойчивости}
\[
F(x_0+\xi) \approx \xi F'(x_0), \quad F'(x_0) = -U''(x_0),
\quad
\text{устойчиво} \Leftrightarrow F'(x_0) < 0.
\]

\textbf{Пример: Пружинный осциллятор}
\[
U(x) = \frac12 k x^2, \quad F = -kx, \quad U''(0) = k > 0.
\]

\textbf{Пример: Математический маятник}
\[
U(\theta) = mgl (1 - \cos \theta), \quad U''(0) = mgl > 0, \quad U''(\pi) = -mgl < 0.
\]

\end{document}