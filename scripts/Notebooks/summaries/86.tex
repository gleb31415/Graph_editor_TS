<latex>
\documentclass[12pt, a4paper]{article}
\usepackage[T2A]{fontenc}
\usepackage[utf8]{inputenc}
\usepackage[russian]{babel}
\usepackage{amsmath, amsfonts, amssymb}
\usepackage{setspace}
\onehalfspacing
\setlength\parindent{0pt}
\usepackage[top=2cm,bottom=2cm,left=1.5cm,right=1.5cm]{geometry}

\begin{document}

\subsubsection*{Логарифм и его свойства}
\[
\log_a b = c \iff a^c = b, \quad a>0, a\neq 1, b>0,
\quad \log_a(bc) = \log_a b + \log_a c.
\]

\subsubsection*{Производная \(a^x\) и число \(e\)}
\[
\frac{d}{dx}a^x = a^x \lim_{h\to0} \frac{a^h - 1}{h} = a^x M(a),
\quad e = \lim_{n\to\infty} \left(1 + \frac{1}{n}\right)^n,
\quad \frac{d}{dx} e^x = e^x.
\]

\subsubsection*{Натуральный логарифм и производные}
\[
y = \ln x \iff e^y = x, \quad \frac{d}{dx} \ln x = \frac{1}{x},
\quad \frac{d}{dx} \log_a x = \frac{1}{x \ln a}.
\]

\subsubsection*{Производные показательных и степенных функций}
\[
\frac{d}{dx} a^x = \ln a \cdot a^x,
\quad \frac{d}{dx} x^p = p x^{p-1}, \quad x>0.
\]

\subsubsection*{Пример логарифмического дифференцирования}
\[
y = x^x, \quad \ln y = x \ln x, \quad y' = x^x (\ln x + 1).
\]

\end{document}
</latex>