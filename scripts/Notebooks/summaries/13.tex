\documentclass[12pt, a4paper]{article}
\usepackage[T2A]{fontenc}
\usepackage[utf8]{inputenc}
\usepackage[russian]{babel}
\usepackage{amsmath}
\usepackage{amssymb}
\usepackage{graphicx}
\usepackage{setspace}
\onehalfspacing
\setlength\parindent{0pt}
\usepackage[top=2cm, bottom=2cm, left=1.5cm, right=1.5cm]{geometry}

\begin{document}

\subsubsection*{Природа силы трения}

Обозначим нормальную реакцию опоры через $N$, а силу трения через $F_{\mathrm{тр}}$.

\subsubsection*{Закон Амонтона-Кулона}

\[
F_{\mathrm{тр},\max} = \mu_s\,N,
\quad
F_{\mathrm{тр}} = \mu_k\,N,
\]

где $\mu_s$ — коэффициент трения покоя, $\mu_k$ — коэффициент трения скольжения.

\textbf{Пример.}

Для блока массы $m$ на горизонтальной поверхности:
\[
F_{\mathrm{тр},\max} = \mu_s mg,
\quad
F_{\mathrm{тр}} = \mu_k mg.
\]

\end{document}