\documentclass[12pt, a4paper]{article}
\usepackage[T2A]{fontenc}
\usepackage[utf8]{inputenc}
\usepackage[russian]{babel}
\usepackage{amsmath, amsfonts, amssymb}
\usepackage{graphicx}
\usepackage{setspace}
\onehalfspacing
\setlength\parindent{0pt}
\usepackage[top=2cm,bottom=2cm,left=1.5cm,right=1.5cm]{geometry}

\begin{document}

\subsubsection*{Теорема о телесном угле}
\[
dE = k\,\frac{\sigma\,dS}{r^2}, \quad k=\frac{1}{4\pi\varepsilon_0},
\quad dE_\perp = k\,\sigma\,d\Omega,
\quad \boxed{E_\perp = k\,\sigma\,\Omega,}
\]
где $\Omega$ — телесный угол.

\textbf{Пример. Бесконечная плоскость} 
\[
E_\perp = \frac{\sigma}{2\varepsilon_0}.
\]

\subsubsection*{Поток электрического поля}
\[
\Phi = \iint \vec E\cdot d\vec S,
\quad \vec E\cdot d\vec S = E\,dS\,\cos\theta.
\]

\textbf{Пример. Сфера с зарядом $q$ в центре}
\[
\Phi_{\text{сфера}} = \frac{q}{\varepsilon_0}.
\]

\subsubsection*{Внешние заряды}
\[
E(r)=k\,\frac{q_{\text{out}}}{r^2}, \quad
|d\Phi_1|=k\,q_{\text{out}}\,d\Omega, \quad
d\Phi_1 + d\Phi_2 = 0.
\]

\subsubsection*{Теорема Гаусса}
\[
\boxed{
\Phi = \iint_{\text{замк.}} \vec E\cdot d\vec S = \frac{1}{\varepsilon_0}\sum q_{\text{in}}.
}
\]

\end{document}