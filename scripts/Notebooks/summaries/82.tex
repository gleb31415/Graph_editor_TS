\documentclass[12pt, a4paper]{article}
\usepackage[T2A]{fontenc}
\usepackage[utf8]{inputenc}
\usepackage[russian]{babel}
\usepackage{amsmath}
\usepackage{graphicx}
\usepackage{setspace}
\onehalfspacing
\setlength\parindent{0pt}
\usepackage[top=2cm,bottom=2cm,left=1.5cm,right=1.5cm]{geometry}

\begin{document}

\subsubsection*{Понятие телесного угла}
\textit{Телесный угол} $\Omega$ — отношение площади $S$ участка на сфере радиуса $r$ к квадрату радиуса:
\[
\Omega=\frac{S}{r^2}.
\]
Полная сфера:
\[
\Omega_{\text{сфера}}=4\pi.
\]

\subsubsection*{Элемент телесного угла и сферические координаты}
В сферических координатах:
\[
d\Omega=\sin\theta\,d\theta\,d\varphi.
\]

\subsubsection*{Конус и сферические сегменты}
Телесный угол конуса с полууглом $\alpha$:
\[
\Omega_{\text{конус}}=2\pi\,(1-\cos\alpha).
\]

\subsubsection*{Практические формулы}
Для удаленного малого объекта площадью $A$ и углом $\beta$:
\[
\Omega\approx \frac{A\cos\beta}{R^2}.
\]

Для круглого диска радиуса $a$ на оси на расстоянии $L$:
\[
\Omega_{\text{диск}}=2\pi\left(1-\frac{L}{\sqrt{L^2+a^2}}\right).
\]

Доля сферы:
\[
f=\frac{\Omega}{4\pi}.
\]

\end{document}