\documentclass[12pt, a4paper]{article}
\usepackage[T2A]{fontenc}
\usepackage[utf8]{inputenc}
\usepackage[russian]{babel}
\usepackage{amsmath}
\usepackage{graphicx}
\usepackage{setspace}
\onehalfspacing
\setlength\parindent{0pt}
\usepackage[top=2cm,bottom=2cm,left=1.5cm,right=1.5cm]{geometry}

\begin{document}

\subsubsection*{Постановка задачи}

В цепях с резисторами и источниками напряжения токи и напряжения определяются системой уравнений Кирхгофа.

\subsubsection*{«Хорошая» симметрия.}

\[
I_1 = I_2 = I.
\]

\[
R_0 = \frac{11}{10}R.
\]

\subsubsection*{«Плохая» симметрия.}

\[
I_1 = -I_2 = I.
\]

\[
R_0 = \frac{6}{5}R.
\]

\subsubsection*{Поворотная симметрия.}

При повороте на $120^\circ$ токи равны.

\end{document}