\documentclass[12pt, a4paper]{article}
\usepackage[T2A]{fontenc}
\usepackage[utf8]{inputenc}
\usepackage[russian]{babel}
\usepackage{amsmath, amsfonts, amssymb}
\usepackage{graphicx}
\usepackage{setspace}
\onehalfspacing
\setlength\parindent{0pt}
\usepackage[top=2cm,bottom=2cm,left=1.5cm,right=1.5cm]{geometry}

\begin{document}

\subsubsection*{Мгновенный центр вращения}

Пусть на теле выбраны две точки $A$ и $B$ с мгновенными скоростями $\vec v_A$ и $\vec v_B$. Пересечение прямых, проведённых через $A$ и $B$ перпендикулярно к $\vec v_A$ и $\vec v_B$, есть МЦВ, точка $O$:
\[
\vec v_O = 0.
\]

Вращение вокруг МЦВ с угловой скоростью
\[
\omega = \frac{|\vec v_A|}{OA},
\]
скорость любой точки $P$ тела
\[
\vec v_P = \vec \omega \times \vec r_P.
\]

\subsubsection*{Переход к ускорениям}

Если $\vec v_A = \vec v_B = \vec 0$, то для ускорений:
\[
(\vec a_A - \vec a_O)\cdot \overrightarrow{OA} = 0, \quad (\vec a_B - \vec a_O)\cdot \overrightarrow{OB} = 0,
\]
откуда $\vec a_O = \vec 0$.

\end{document}