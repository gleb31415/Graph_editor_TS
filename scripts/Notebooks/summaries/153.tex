\documentclass[12pt, a4paper]{article}
\usepackage[T2A]{fontenc}
\usepackage[utf8]{inputenc}
\usepackage[russian]{babel}
\usepackage{amsmath, amsfonts, amssymb}
\usepackage{graphicx}
\usepackage{setspace}
\onehalfspacing
\setlength\parindent{0pt}
\usepackage[top=2cm, bottom=2cm, left=1.5cm, right=1.5cm]{geometry}

\begin{document}

\subsubsection*{Энергия системы зарядов}
\[
U=\frac12\sum_{i=1}^N q_i\,\varphi(A_i),
\]
где \(\varphi(A_i)\) — потенциал в точке \(A_i\), созданный всеми остальными зарядами.

\subsubsection*{Заряд \(q\) и проводящая пластина}

\begin{center}
\includegraphics[width=0.33\linewidth]{10image1.jpeg}
\end{center}

\[
\varphi_{\text{пл}}(A_q)=-\,\frac{k\,q}{2H},\quad k=\frac{1}{4\pi\varepsilon_0}.
\]

\[
\boxed{
U_{q\leftrightarrow \text{пл}}=-\,\frac{k\,q^2}{2H},\quad
U_{\text{сист}}=-\,\frac{k\,q^2}{4H},\quad
U_{\text{пл}\leftrightarrow \text{пл}}=\frac{k\,q^2}{4H}.
}
\]

\end{document}