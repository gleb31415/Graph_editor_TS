\documentclass[12pt, a4paper]{article}
\usepackage[T2A]{fontenc}
\usepackage[utf8]{inputenc}
\usepackage[russian]{babel}
\usepackage{amsmath}
\usepackage{amssymb}
\usepackage{setspace}
\onehalfspacing
\setlength\parindent{0pt}
\usepackage[top=2cm,bottom=2cm,left=1.5cm,right=1.5cm]{geometry}

\begin{document}

\subsubsection*{Определение силы и третий закон Ньютона}
\textit{Сила} обозначается вектором $\vec F$.  
\textit{Третий закон Ньютона}:
\[
\vec F_{A\to B} = -\,\vec F_{B\to A}.
\]

\subsubsection*{Равнодействующая сила}
\[
\vec F_{\rm равн} = \sum_i \vec F_i.
\]

\subsubsection*{Условие покоя тела}
\[
\sum_i \vec F_i = 0.
\]

\end{document}