\documentclass[12pt, a4paper]{article}
\usepackage[T2A]{fontenc}
\usepackage[utf8]{inputenc}
\usepackage[russian]{babel}
\usepackage{amsmath}
\usepackage{graphicx}
\usepackage{setspace}
\onehalfspacing
\setlength\parindent{0pt}
\usepackage[top=2cm,bottom=2cm,left=1.5cm,right=1.5cm]{geometry}

\begin{document}

\subsubsection*{Понятие ВАХ}

\textit{Вольт–амперная характеристика (ВАХ)} — график зависимости тока $I$ от напряжения $U$.

\subsubsection*{Последовательное соединение}

\[
U = U_1 + U_2, \quad I = f_1(U_1) = f_2(U_2).
\]

\subsubsection*{Параллельное соединение}

\[
I = I_1 + I_2 = f_1(U) + f_2(U).
\]

\subsubsection*{Нагрузочная прямая}

Для источника ЭДС $\mathscr{E}$ с внутренним сопротивлением $R$:

\[
U = \mathscr{E} - IR, \quad I = \frac{\mathscr{E} - U}{R}.
\]

Рабочая точка определяется решением системы

\[
\begin{cases}
I = I_{\rm эл}(U), \\
U = \mathscr{E} - IR.
\end{cases}
\]

\subsubsection*{Диоды}

Идеальный диод: ток в прямом направлении без падения напряжения, в обратном — блокировка.

\end{document}