<latex>
\documentclass[12pt, a4paper]{article}
\usepackage[T2A]{fontenc}
\usepackage[utf8]{inputenc}
\usepackage[russian]{babel}
\usepackage{amsmath}
\usepackage{amssymb}
\usepackage{setspace}
\onehalfspacing
\setlength\parindent{0pt}
\usepackage[top=2cm,bottom=2cm,left=1.5cm,right=1.5cm]{geometry}

\begin{document}

\subsubsection*{Важные понятия термодинамики}
\textit{Внутренняя энергия} $U$ — сумма кинетической и потенциальной энергии частиц.

\textit{Количество теплоты} $Q$ — энергия, переданная телу или отданная им.

\subsubsection*{Нагревание и охлаждение}
\[
Q = cm(T_2 - T_1),
\]
где $c$ — удельная теплоёмкость, $m$ — масса, $T_2 - T_1$ — изменение температуры.

\subsubsection*{Фазовые переходы}
\textit{Удельная теплота плавления} $\lambda$:
\[
Q = \lambda m.
\]

\textit{Удельная теплота кипения} $L$:
\[
Q = L m.
\]

\textbf{Пример.} Нагрев и плавление:
\[
Q = cm(T_0 - T_1) + \lambda m.
\]

\end{document}
</latex>