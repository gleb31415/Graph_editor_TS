\documentclass[12pt, a4paper]{article}
\usepackage[T2A]{fontenc}
\usepackage[utf8]{inputenc}
\usepackage[russian]{babel}
\usepackage{amsmath, amsfonts, amssymb}
\usepackage{graphicx}
\usepackage{setspace}
\onehalfspacing
\setlength\parindent{0pt}
\usepackage[top=2cm,bottom=2cm,left=1.5cm,right=1.5cm]{geometry}

\begin{document}

\subsubsection*{Метод изображений: точечный заряд и проводящая плоскость}

Пусть плоскость $z=0$ с потенциалом $\varphi=0$, заряд $q$ в точке $\vec r_0=(0,0,H)$, изображение $-q$ в $\vec r_0'=(0,0,-H)$. Тогда потенциал при $z>0$:
\[
\varphi(\vec r)=k\left(\frac{q}{|\vec r-\vec r_0|}-\frac{q}{|\vec r-\vec r_0'|}\right), \quad k=\frac{1}{4\pi\varepsilon_0}.
\]

Индуцированный заряд на плоскости:
\[
\sigma(\rho)=-\frac{q\,H}{2\pi(\rho^2+H^2)^{3/2}}, \quad \int \sigma\, dS = -q.
\]

Сила притяжения:
\[
F=\frac{k\,q^2}{4H^2}.
\]

\subsubsection*{Изображение в сфере}

Проводящая сфера радиуса $R$, заряд $q$ на расстоянии $d>R$ от центра. Изображение:
\[
q' = -q \frac{R}{d}, \quad r' = \frac{R^2}{d}.
\]

Потенциал вне сферы ($r \ge R$):
\[
\varphi(\vec r) = k\left(\frac{q}{|\vec r - \vec d|} + \frac{q'}{|\vec r - \vec r'|}\right).
\]

Индуцированный заряд и сила:
\[
Q_{\text{ин}} = q' = -q \frac{R}{d}, \quad F = \frac{k q^2 R d}{(d^2 - R^2)^2}.
\]

\end{document}