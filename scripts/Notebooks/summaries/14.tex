\documentclass[12pt, a4paper]{article}
\usepackage[T2A]{fontenc}
\usepackage[utf8]{inputenc}
\usepackage[russian]{babel}
\usepackage{amsmath}
\usepackage{graphicx}
\setlength\parindent{0pt}
\usepackage[top=2cm,bottom=2cm,left=1.5cm,right=1.5cm]{geometry}

\begin{document}

\subsubsection*{Момент силы}

\[
M_O = Fr.
\]

\subsubsection*{Условие равновесия}

\[
\sum_i F_i = 0
\quad\text{и}\quad
\sum_i M_{O,i} = \sum_i F_i r_i = 0.
\]

\[
F_1 a = F_2 b.
\]

\subsubsection*{Теорема о трёх силах}

Если на тело действуют три непараллельные силы \(F_1, F_2, F_3\), то равновесие тогда и только тогда, когда их линии действия пересекаются в одной точке \(O\).

\end{document}