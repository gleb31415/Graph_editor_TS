\documentclass[12pt, a4paper]{article}
\usepackage[T2A]{fontenc}
\usepackage[utf8]{inputenc}
\usepackage[russian]{babel}
\usepackage{amsmath}
\usepackage{graphicx}
\setlength\parindent{0pt}
\usepackage[top=2cm,bottom=2cm,left=1.5cm,right=1.5cm]{geometry}

\begin{document}
\subsubsection*{Полное внутреннее отражение: условие и критический угол}
Пусть \(n_1 > n_2\). Закон Снеллиуса:
\[
n_1 \sin \theta_1 = n_2 \sin \theta_2, \quad 0 \le \theta_2 \le 90^\circ.
\]

Граничный угол падения \(\theta_c\):
\[
\sin \theta_c = \frac{n_2}{n_1}.
\]

Если \(\theta_1 > \theta_c\), возникает полное внутреннее отражение.

\textbf{Пример:} Для алмаза \(n \approx 2.4\), \(\theta_c \approx \arcsin(1/2.4) \approx 24.6^\circ\).

\textbf{Условие видимости монетки:}
\[
90^\circ - \theta_c < \theta_c \quad \Longrightarrow \quad \theta_c > 45^\circ, \quad n < \sqrt{2}.
\]

\end{document}