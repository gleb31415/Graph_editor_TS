<latex>
\documentclass[12 pt, a4paper]{article}
\usepackage[T2A]{fontenc}
\usepackage[utf8]{inputenc}
\usepackage[russian]{babel}
\usepackage{amsmath}
\usepackage{amssymb}
\usepackage{setspace}
\onehalfspacing
\usepackage{geometry}
\setlength\parindent{0pt}
\usepackage[top=2cm,bottom=2cm,left=1.5cm,right=1.5cm]{geometry}

\begin{document}

Масса обозначается $m$, измеряется в килограммах (кг).

\textit{Плотность} обозначается $\rho$ и определяется формулой:
\[
\rho = \frac{m}{V},
\]
где $m$ — масса, $V$ — объём.

Средняя плотность тела с массой $m_{\text{всего}}$ и объёмом $V_{\text{всего}}$:
\[
\rho_{\text{ср}} = \frac{m_{\text{всего}}}{V_{\text{всего}}}.
\]

\textbf{Пример.} Для двух тел с массами $m_1$, $m_2$ и объёмами $V_1$, $V_2$:
\[
\rho_{\text{ср}} = \frac{m_1 + m_2}{V_1 + V_2}.
\]

\end{document}
</latex>