\documentclass[12pt, a4paper]{article}
\usepackage[T2A]{fontenc}
\usepackage[utf8]{inputenc}
\usepackage[russian]{babel}
\usepackage{amsmath}
\usepackage{amssymb}
\usepackage{graphicx}
\usepackage{setspace}
\onehalfspacing
\setlength\parindent{0pt}
\usepackage[top=2cm,bottom=2cm,left=1.5cm,right=1.5cm]{geometry}

\begin{document}

\subsubsection*{Связь косинуса и тангенса}
\[
\sin^2\alpha + \cos^2\alpha = 1,
\quad
\tan^2\alpha + 1 = \frac{1}{\cos^2\alpha}.
\]

\subsubsection*{Площадь треугольника}
\[
S_\Delta = \frac{ab\sin\alpha}{2}.
\]

\subsubsection*{Формулы суммы и разности углов}
\[
\sin(\alpha + \beta) = \sin\alpha\cos\beta + \cos\alpha\sin\beta,
\quad
\cos(\alpha + \beta) = \cos\alpha\cos\beta - \sin\alpha\sin\beta.
\]

\subsubsection*{Формулы двойных углов}
\[
\cos2\alpha = \cos^2\alpha - \sin^2\alpha,
\quad
\sin2\alpha = 2\sin\alpha\cos\alpha.
\]

\subsubsection*{Теорема синусов}
\[
\frac{a}{\sin A} = \frac{b}{\sin B} = \frac{c}{\sin C}.
\]

\subsubsection*{Теорема косинусов}
\[
a^2 = b^2 + c^2 - 2bc\cos A.
\]

\end{document}