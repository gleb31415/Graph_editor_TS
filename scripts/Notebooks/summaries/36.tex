\documentclass[12pt, a4paper]{article}
\usepackage[T2A]{fontenc}
\usepackage[utf8]{inputenc}
\usepackage[russian]{babel}
\usepackage{amsmath}
\usepackage{graphicx}
\usepackage{setspace}
\onehalfspacing
\setlength\parindent{0pt}
\usepackage[top=2cm,bottom=2cm,left=1.5cm,right=1.5cm]{geometry}

\begin{document}

\subsubsection*{Почему при равномерном движении есть ускорение}

\[
\vec a = \frac{\Delta\vec v}{\Delta t}.
\]

\subsubsection*{Центростремительное ускорение}

\[
a \approx \frac{v^2}{r} = \omega v = \omega^2 r.
\]

\end{document}