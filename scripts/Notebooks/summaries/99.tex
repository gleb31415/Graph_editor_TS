\documentclass[12pt, a4paper]{article}
\usepackage[T2A]{fontenc}
\usepackage[utf8]{inputenc}
\usepackage[russian]{babel}
\usepackage{amsmath}
\usepackage{graphicx}
\usepackage{setspace}
\onehalfspacing
\setlength\parindent{0pt}
\usepackage[top=2cm,bottom=2cm,left=1.5cm,right=1.5cm]{geometry}

\begin{document}

\subsubsection*{Процесс наложения цепей}

\begin{center}
\includegraphics[width=0.25\linewidth]{8ov1.png}
\end{center}

\subsubsection*{Нахождение сопротивления цепи с помощью метода наложения}

\begin{center}
\includegraphics[width=0.7\linewidth]{8ov2.png}
\end{center}

\[
R_0 = \frac{U_\text{общ}}{I_\text{общ}} = \frac{17}{24}R.
\]

\subsubsection*{Наложение в бесконечных сетках}

\begin{center}
\includegraphics[width=0.43\linewidth]{8ov4.png}
\end{center}

\begin{center}
\includegraphics[width=0.85\linewidth]{8overlap5.png}
\end{center}

\[
R_0 = \frac{U_\text{общ}}{I_\text{общ}} = \frac{R}{2}.
\]

\end{document}