\documentclass[12 pt, a4paper]{article}% тип документа, размер шрифта
\usepackage[T2A]{fontenc}%поддержка кириллицы в ЛаТеХ
\usepackage[utf8]{inputenc}%кодировка
\usepackage[russian]{babel}%русский язык
\usepackage{mathtext}% русский текст в формулах
\usepackage{amsmath}%удобная вёрстка многострочных формул, масштабирующийся текст в формулах, формулы в рамках и др.
\usepackage{amsfonts}%поддержка ажурного и готического шрифтов — например, для записи символа {\displaystyle \mathbb {R} } \mathbb {R} 
\usepackage{amssymb}%amsfonts + несколько сотен дополнительных математических символов
\frenchspacing%запрет длинного пробела после точки
\usepackage{setspace}%возможность установки межстрочного интервала
\usepackage{indentfirst}%пакет позволяет делать в первом абзаце после заголовка абзацный отступ
\usepackage[unicode, pdftex]{hyperref}
\onehalfspacing%установка полуторного интервала по умолчанию
\usepackage{graphicx}%подключение рисунков
\graphicspath{{images/}}%путь ко всем рисункам
\usepackage{caption}
\usepackage{float}%плавающие картинки
\usepackage{tikz} % это для чудо-миллиметровки
\usepackage{pgfplots}%для построения графиков
\pgfplotsset{compat=newest, y label style={rotate=-90},  width=10 cm}%версия пакета построения графиков, ширина графиков
\usepackage{pgfplotstable}%простое рисование табличек
\usepackage{lastpage}%пакет нумерации страниц
\usepackage{comment}%возможность вставлять большие комменты
\usepackage{float}
%%%%% ПОЛЯъ
\setlength\parindent{0pt} 
\usepackage[top = 2 cm, bottom = 2 cm, left = 1.5 cm, right = 1.5 cm]{geometry}
\setlength\parindent{0pt}
%%%%% КОЛОНТИТУЛЫ
\usepackage{xcolor}
\usepackage{amsmath}
\usepackage{gensymb}
\usepackage{tikz}
\newcommand\spiral{}% Just for safety so \def won't overwrite something
\def\spiral[#1](#2)(#3:#4:#5){% \spiral[draw options](placement)(end angle:revolutions:final radius)
\pgfmathsetmacro{\domain}{pi*#3/180+#4*2*pi}
\draw [#1,shift={(#2)}, domain=0:\domain,variable=\t,smooth,samples=int(\domain/0.08)] plot ({\t r}: {#5*\t/\domain})
}


\begin{document}
Начнём с пары определений, которые позволят нам лучше понять определение массы:

Инерция — это явление сохранения состояния покоя или равномерного прямолинейного движения при отсутствии внешних воздействий.

Инертность — это свойство тела сопротивляться изменению скорости при внешнем воздействии. 

Масса — это физическая величина, являющаяся мерой инертности тела. Таким образом, масса численно выражает, насколько сильно тело сопротивляется изменению своего движения, то есть насколько оно инертно. 
Масса также характеризует количество вещества в нём (например, если прилепить к куску пластилина ещё кусок, масса куска увеличится). Масса обозначается $m$, измеряется в килограммах (кг).

\textit{Плотность} обозначается буквой $\rho$ и показывает, сколько массы приходится на единицу объёма. Если взять кубик объёмом $1$ м$^3$ и заполнить его веществом, то масса этого кубика и будет численно равна плотности. Формульное определение плотности:
\[
\rho = \frac{m}{V},
\]
где $m$ — масса тела, $V$ — его объём, а $\rho$ — плотность. Измеряется в кг/м$^3$. Чем больше $\rho$, тем «тяжелее» единица объёма этого вещества. Важный момент: если взять из вещества кусочек, оторвать, сжать, переложить — его плотность не изменится, если само вещество не поменялось. То есть плотность — это свойство вещества, а не формы тела.

Иногда тело состоит из нескольких частей с разной плотностью, или просто имеет пустоты (например, шар из пенопласта). Тогда говорят про среднюю плотность — это когда мы берём \textit{всю массу тела} и \textit{делим на весь объём}:
\[
\rho_{\text{ср}} = \frac{m_{\text{всего}}}{V_{\text{всего}}}.
\]
Даже если внутри плотность везде разная, средняя — это то, как будто всё вещество распределено равномерно по объёму. Этой формулой можно пользоваться, даже если мы вообще не знаем, из чего состоит тело, лишь бы у нас были масса и объём.

\textbf{Пример 1.} Даны два тела: первое с массой $m_1$ и объёмом $V_1$, второе с массой $m_2$ и объёмом $V_2$. Тогда их общая масса $m_{\text{общ}} = m_1 + m_2$, общий объём $V_{\text{общ}} = V_1 + V_2$, и средняя плотность
\[
\rho_{\text{ср}} = \frac{m_{\text{общ}}}{V_{\text{общ}}}
= \frac{m_1 + m_2}{V_1 + V_2}.
\]

\textbf{Пример 2.} Два тела имеют одинаковую массу $m$, но плотности разные: у первого $\rho$, у второго $2\rho$. Объёмы
\[
V_1 = \frac{m}{\rho},\quad V_2 = \frac{m}{2\rho},
\]
общее $V_{\text{общ}} = \frac{m}{\rho} + \frac{m}{2\rho} = \frac{3m}{2\rho}$, общая масса $2m$, поэтому
\[
\rho_{\text{ср}} = \frac{2m}{\;3m/(2\rho)\;} = \frac{2m\cdot2\rho}{3m} = \frac{4\rho}{3}.
\]


\end{document}