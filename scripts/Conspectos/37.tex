\documentclass[12pt, a4paper]{article}% тип документа, размер шрифта
\usepackage[T2A]{fontenc}%поддержка кириллицы в ЛаТеХ
\usepackage[utf8]{inputenc}%кодировка
\usepackage[russian]{babel}%русский язык
\usepackage{mathtext}% русский текст в формулах
\usepackage{amsmath}%удобная вёрстка многострочных формул, масштабирующийся текст в формулах, формулы в рамках и др.
\usepackage{amsfonts}%поддержка ажурного и готического шрифтов — например, для записи символа {\displaystyle \mathbb {R} } \mathbb {R} 
\usepackage{amssymb}%amsfonts + несколько сотен дополнительных математических символов
\frenchspacing%запрет длинного пробела после точки
\usepackage{setspace}%возможность установки межстрочного интервала
\usepackage{indentfirst}%пакет позволяет делать в первом абзаце после заголовка абзацный отступ
\usepackage[unicode, pdftex]{hyperref}
\onehalfspacing%установка полуторного интервала по умолчанию
\usepackage{graphicx}%подключение рисунков
\graphicspath{{images/}}%путь ко всем рисункам
\usepackage{caption}
\usepackage{float}%плавающие картинки
\usepackage{tikz} % это для чудо-миллиметровки
\usepackage{pgfplots}%для построения графиков
\pgfplotsset{compat=newest, y label style={rotate=-90},  width=10 cm}%версия пакета построения графиков, ширина графиков
\usepackage{pgfplotstable}%простое рисование табличек
\usepackage{lastpage}%пакет нумерации страниц
\usepackage{comment}%возможность вставлять большие комменты
\usepackage{float}
%%%%% ПОЛЯъ
\setlength\parindent{0pt} 
\usepackage[top = 2 cm, bottom = 2 cm, left = 1.5 cm, right = 1.5 cm]{geometry}
\setlength\parindent{0pt}
%%%%% КОЛОНТИТУЛЫ
\usepackage{xcolor}
\usepackage{amsmath}
\usepackage{gensymb}
\usepackage{tikz}

\begin{document}



\subsubsection*{Тангенциальное и нормальное ускорения}


\begin{center}
\includegraphics[width=0.4\linewidth]{9alpha2.png}
\label{fig:mpr}
\end{center}

При неравномерном движении по окружности скорость по модулю $v$ меняется, и ускорение разбивается на две ортогональные составляющие:
\[
\vec a = \vec a_\tau + \vec a_n,
\]
где 
\begin{itemize}
	\item \(\vec a_\tau\) — \textit{тангенциальное ускорение} (по касательной к траектории, изменяет модуль скорости);
	\item \(\vec a_n\) — \textit{нормальное ускорение} (вдоль радиуса, изменяет направление).
\end{itemize}

\subsubsection*{Нормальное ускорение}
При любом движении по окружности радиуса $r$ нормальная составляющая всегда равна
\[
a_n = \frac{v^2}{r} = \omega v,
\]
и направлена к центру окружности.

\subsubsection*{Угловое ускорение и угол поворота}
Угловое ускорение $\alpha$ определяет изменение угловой скорости:
\[
\alpha = \frac{d\omega}{dt}.
\]
При постоянном $\alpha$:
\[
\omega(t) = \omega_0 + \alpha\,t,
\]

По аналогии с равноускоренным прямолинейным движением, 
\[
\Delta \varphi(t) = \omega_0t + \tfrac12\alpha\,t^2.
\]

\subsubsection*{Траектория произвольного движения}
Для любой криволинейной траектории ускорение тоже раскладывается на нормальную и тангенциальную составляющие:
\[
\vec a = \underbrace{\dot v}_{a_\tau}\,\vec e_\tau + \underbrace{\frac{v^2}{R}}_{a_n}\,\vec e_n,
\]
где $\dot v = \dfrac{\Delta |\vec v|}{\Delta t}$ при $\Delta t \to 0$, $R$ — \textit{радиус кривизны} в данной точке, $\vec e_\tau$ — касательный единичный вектор, $\vec e_n$ — нормальный (кривая локально приближается окружностью радиуса $R$).




Радиус $R$ кривизны траектории определяется как 
\[
R = \frac{v^2}{a_n}.
\]

\end{document}