\documentclass[12pt, a4paper]{article}% тип документа, размер шрифта
\usepackage[T2A]{fontenc}%поддержка кириллицы в ЛаТеХ
\usepackage[utf8]{inputenc}%кодировка
\usepackage[russian]{babel}%русский язык
\usepackage{mathtext}% русский текст в формулах
\usepackage{amsmath}%удобная вёрстка многострочных формул, масштабирующийся текст в формулах, формулы в рамках и др.
\usepackage{amsfonts}%поддержка ажурного и готического шрифтов — например, для записи символа {\displaystyle \mathbb {R} } \mathbb {R} 
\usepackage{amssymb}%amsfonts + несколько сотен дополнительных математических символов
\frenchspacing%запрет длинного пробела после точки
\usepackage{setspace}%возможность установки межстрочного интервала
\usepackage{indentfirst}%пакет позволяет делать в первом абзаце после заголовка абзацный отступ
\usepackage[unicode, pdftex]{hyperref}
\onehalfspacing%установка полуторного интервала по умолчанию
\usepackage{graphicx}%подключение рисунков
\graphicspath{{images/}}%путь ко всем рисункам 
\usepackage{caption}
\usepackage{float}%плавающие картинки
\usepackage{tikz} % это для чудо-миллиметровки
\usepackage{pgfplots}%для построения графиков
\pgfplotsset{compat=newest, y label style={rotate=-90},  width=10 cm}%версия пакета построения графиков, ширина графиков
\usepackage{pgfplotstable}%простое рисование табличек
\usepackage{lastpage}%пакет нумерации страниц
\usepackage{comment}%возможность вставлять большие комменты
\usepackage{float}
%%%%% ПОЛЯъ
\setlength\parindent{0pt} 
\usepackage[top = 2 cm, bottom = 2 cm, left = 1.5 cm, right = 1.5 cm]{geometry}
\setlength\parindent{0pt}
%%%%% КОЛОНТИТУЛЫ
\usepackage{xcolor}
\usepackage{amsmath}
\usepackage{gensymb}
\usepackage{tikz}

\begin{document}


\subsubsection*{Связь косинуса и тангенса}
Синус и косинус связаны соотношением  
\[
\sin^2\alpha + \cos^2\alpha = 1.
\]
Из него вытекает  
\[
\tan^2\alpha + 1 = \frac{\sin^2\alpha}{\cos^2\alpha} + 1 = \frac{\sin^2\alpha + \cos^2\alpha}{\cos^2\alpha} = \frac{1}{\cos^2\alpha}.
\]

Аналогично можно вывести

\[
\frac{1}{\sin^2\alpha} = \cot^2\alpha + 1
\]

\subsubsection*{Площадь треугольника}

Мы знаем простую формулу площади треугольника через длину высоты и основания:

\[
S_\Delta = \frac{ah}{2}.
\]

Мы также можем выразить высоту $h$ треугольника через одну из сторон с помощью тригонометрических функций.

\begin{center}
\includegraphics[width=0.33\linewidth]{9trig1.jpeg}
\label{fig:mpr}
\end{center}

\[
h = b\sin\alpha 
\quad\Longrightarrow\quad
S_\Delta = \frac{ab\sin\alpha}{2}
\]

Таким образом, мы получили формулу площади треугольника через две стороны и угол между ними. Понятно, что площадь параллелограмма аналогично можно выразить через его стороны и угол:

\[
S_\Delta = ab\sin\alpha.
\]


\subsubsection*{Формулы суммы и разности углов}
Пусть у нас есть два угла $\alpha$ и $\beta$. Тогда, рассматривая поворот вектора на угол $\alpha$ а затем на $\beta$, получаем

\[
\sin(\alpha + \beta) = \sin\alpha\cos\beta + \cos\alpha\sin\beta.
\]


Докажем эту формулу. Для этого возьмём прямоугольник на рисунке, разбитый на $4$ 
прямоугольных треугольника с гипотенузой $1$, два из которых с углом $\alpha$, два других
--- с углом $\beta$, и $2$ прямоугольника со сторонами, выражающимияся через тригонометрические
функции $\alpha$ и $\beta$:

\begin{center}
\includegraphics[width=0.43\linewidth]{9trig2.jpeg}
\label{fig:mpr}
\end{center}

Тот же прямоугольник можно разбить на те же $4$ треугольника, расставленных иначе, 
и ромб со стороной $1$ и углом $\alpha+\beta$:

\begin{center}
\includegraphics[width=0.43\linewidth]{9trig3.jpeg}
\label{fig:mpr}
\end{center}

Площадь ромба и сумма площадей двух прямоугольников с прошлого рисунка равны, то есть

\[
1\cdot 1\cdot\sin(\alpha+\beta) = \sin\alpha\cdot\cos\beta+\cos\alpha\cdot\sin\beta,
\]

что и требовалось доказать. Чтобы получить косинус суммы углов, просто воспользуемся свойством $\sin(90^\circ-\alpha) = \cos\alpha$.


\[
\cos(\alpha + \beta) = \sin(90^\circ-\alpha-\beta) = \sin(90^\circ-\alpha)\cos(-\beta) + \cos(90^\circ-\alpha)\sin(-\beta) =  \cos\alpha\cos\beta - \sin\alpha\sin\beta,
\]

Для разности углов достаточно заменить $\beta$ на $-\beta$, учтя что $\cos(-\beta)=\cos\beta$ и $\sin(-\beta)=-\sin\beta$:
\[
\cos(\alpha - \beta) = \cos\alpha\cos\beta + \sin\alpha\sin\beta,
\]
\[
\sin(\alpha - \beta) = \sin\alpha\cos\beta - \cos\alpha\sin\beta.
\]
Отсюда напрямую вытекают обратные преобразования (формулы для произведения тригонометрических функций):
\[
\sin A \,\sin B = \tfrac{1}{2}\bigl[\cos(A - B) - \cos(A + B)\bigr],
\]
\[
\cos A \,\cos B = \tfrac{1}{2}\bigl[\cos(A - B) + \cos(A + B)\bigr],
\]
\[
\sin A \,\cos B = \tfrac{1}{2}\bigl[\sin(A + B) + \sin(A - B)\bigr],
\]

а также формулы преобразования суммы и разности тригонометрических функций:

\[
\sin \alpha + \sin \beta = 2 \sin\,\!\bigl(\tfrac{\alpha + \beta}{2}\bigr)\,\cos\,\!\bigl(\tfrac{\alpha - \beta}{2}\bigr),
\]
\[
\sin \alpha - \sin \beta = 2 \cos\,\!\bigl(\tfrac{\alpha + \beta}{2}\bigr)\,\sin\,\!\bigl(\tfrac{\alpha - \beta}{2}\bigr),
\]
\[
\cos \alpha + \cos \beta = 2 \cos\,\!\bigl(\tfrac{\alpha + \beta}{2}\bigr)\,\cos\,\!\bigl(\tfrac{\alpha - \beta}{2}\bigr),
\]
\[
\cos \alpha - \cos \beta = -2 \sin\,\!\bigl(\tfrac{\alpha + \beta}{2}\bigr)\,\sin\,\!\bigl(\tfrac{\alpha - \beta}{2}\bigr).
\]


\subsubsection*{Формулы двойных и половинных углов}
Двойные углы получаются из формул суммы при $\alpha=\beta$:
\[
\cos2\alpha = \cos^2\alpha - \sin^2\alpha = 2\cos^2\alpha - 1 = 1 - 2\sin^2\alpha,
\]
\[
\sin2\alpha = 2\sin\alpha\cos\alpha.
\]
Половинные углы выводятся из формулы для $\cos2\alpha$, решая относительно $\alpha$:
\[
\cos^2\alpha = \frac{1 + \cos2\alpha}{2},\quad \sin^2\alpha = \frac{1 - \cos2\alpha}{2}.
\]

\subsubsection*{Обратные тригонометрические функции}
Арксинус, арккосинус, арктангенс — обратные функции синуса, косинуса и тангенса. Например, если $\sin\varphi = x$ и $-1\le x\le1$, то
\[
\varphi = \arcsin x,\quad -\frac{\pi}{2}\le \varphi\le\frac{\pi}{2}.
\]
Аналогично $\varphi = \arccos x$ при $0\le\varphi\le\pi$ и $\varphi = \arctan x$ при $-\frac{\pi}{2}<\varphi<\frac{\pi}{2}$.

\subsubsection*{Теорема синусов}
В произвольном треугольнике \(ABC\) обозначим стороны \(a=BC\), \(b=CA\), \(c=AB\),
а углы при вершинах соответственно \(A,B,C\).


\begin{center}
\includegraphics[width=0.45\linewidth]{9trig4.jpeg}
\label{fig:mpr}
\end{center}



Площадь \(S\) этого треугольника можно выразить через любую пару сторон и синус угла между ними:
\[
S = \frac12 bc\sin A = \frac12 a c\sin B = \frac12 a b\sin C.
\]
Домножим каждое равенство на \(2/c\), \(2/a\), \(2/b\) и получим
\[
\frac{a}{\sin A} = \frac{b}{\sin B} = \frac{c}{\sin C}.
\]

\subsubsection*{Теорема косинусов}
В том же треугольнике опустим высоту из вершины \(C\) на сторону \(c=AB\). Пусть проекция \(A\)–\(C\) на \(AB\) делит \(c\) на части \(x\) и \(c-x\).

\begin{center}
\includegraphics[width=0.45\linewidth]{9trig5.jpeg}
\label{fig:mpr}
\end{center}



По теореме Пифагора
\[
b^2 = h^2 + x^2,\quad a^2 = h^2 + (c-x)^2,
\]
где \(h\) — высота. Вычитая первое из второго:
\[
a^2 - b^2 = (c-x)^2 - x^2 = c^2 - 2cx,
\]
то есть \(x = \dfrac{c^2 + b^2 - a^2}{2c}\). Но также \(b\cos A = x\), следовательно
\[
b\cos A = \frac{b^2 + c^2 - a^2}{2c}
\;\Longrightarrow\;
a^2 = b^2 + c^2 - 2bc\cos A.
\]
Аналогично получаются формулы
\[
c^2 = a^2 + b^2 - 2ab\cos C,\quad
b^2 = a^2 + c^2 - 2ac\cos B.
\]










\end{document}