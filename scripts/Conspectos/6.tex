\documentclass[12pt, a4paper]{article}% тип документа, размер шрифта
\usepackage[T2A]{fontenc}%поддержка кириллицы в ЛаТеХ
\usepackage[utf8]{inputenc}%кодировка
\usepackage[russian]{babel}%русский язык
\usepackage{mathtext}% русский текст в формулах
\usepackage{amsmath}%удобная вёрстка многострочных формул, масштабирующийся текст в формулах, формулы в рамках и др.
\usepackage{amsfonts}%поддержка ажурного и готического шрифтов — например, для записи символа {\displaystyle \mathbb {R} } \mathbb {R} 
\usepackage{amssymb}%amsfonts + несколько сотен дополнительных математических символов
\frenchspacing%запрет длинного пробела после точки
\usepackage{setspace}%возможность установки межстрочного интервала
\usepackage{indentfirst}%пакет позволяет делать в первом абзаце после заголовка абзацный отступ
\usepackage[unicode, pdftex]{hyperref}
\onehalfspacing%установка полуторного интервала по умолчанию
\usepackage{graphicx}%подключение рисунков
\graphicspath{{images/}}%путь ко всем рисункам
\usepackage{caption}
\usepackage{float}%плавающие картинки
\usepackage{tikz} % это для чудо-миллиметровки
\usepackage{pgfplots}%для построения графиков
\pgfplotsset{compat=newest, y label style={rotate=-90},  width=10 cm}%версия пакета построения графиков, ширина графиков
\usepackage{pgfplotstable}%простое рисование табличек
\usepackage{lastpage}%пакет нумерации страниц
\usepackage{comment}%возможность вставлять большие комменты
\usepackage{float}
%%%%% ПОЛЯъ
\setlength\parindent{0pt} 
\usepackage[top = 2 cm, bottom = 2 cm, left = 1.5 cm, right = 1.5 cm]{geometry}
\setlength\parindent{0pt}
%%%%% КОЛОНТИТУЛЫ
\usepackage{xcolor}
\usepackage{amsmath}
\usepackage{gensymb}
\usepackage{tikz}

\begin{document}

\subsubsection*{Понятие температуры}

Начнём с самого простого понимания: температура — это не «горячо» или «холодно» абстрактно, а мера того, как быстро движутся молекулы вещества. 
Любое вещество состоит из очень маленьких молекул, которые беспорядочно движутся с большими скоростями внутри сосуда. Чем быстрее они движутся, тем выше температура.

В кинетической теории газа вводят понятие средней кинетической энергии одной молекулы:
\[
\langle E_{\text{к}} \rangle = \frac12 m \langle v^2 \rangle,
\]
где
\begin{itemize}
  \item $m$ — масса одной молекулы;
  \item $\langle v^2 \rangle$ — средний квадрат скорости молекулы.
\end{itemize}

\subsubsection*{Связь с кинетической энергией}
Соотношение между средней кинетической энергией молекул и абсолютной температурой $T$ (в кельвинах) имеет вид
\[
\langle E_{\text{к}}\rangle = \frac32 k T,
\]
где
\begin{itemize}
  \item $\langle E_{\text{к}}\rangle$ — средняя кинетическая энергия одной молекулы (в джоулях);
  \item $k$ — постоянная Больцмана, численно $1{.}38\times10^{-23}$ Дж/К;
  \item $T$ — абсолютная температура (в кельвинах).
\end{itemize}
Эта формула показывает, что температура прямо пропорциональна средней кинетической энергии хаотического движения молекул.

Отсюда можно выразить температуру через энергию:
\[
T = \frac{2}{3k}\,\langle E_{\text{к}}\rangle.
\]
То есть, чтобы повысить температуру вещества, надо увеличить среднюю энергию его молекул: например, нагреть стенки сосуда с газом, чтобы они передавали молекулам газа больше энергии при соударениях.

Абсолютная нулевая точка ($T = 0$ К) соответствует ситуации, когда молекулы полностью прекратили тепловое движение и их средняя кинетическая энергия равна нулю. На практике до $0$ К добраться невозможно, но приближение к этой температуре используется в криогенной технике.

Переход от Цельсия к Кельвину: температура в Цельсиях $t$ связана с абсолютной температурой $T$ так:
\[
T = t + 273{.}15.
\]
Например, если на градуснике стоит $27^\circ$C, это означает, что молекулы обладают средней кинетической энергией, соответствующей $T=300{.}15$ К:
\[
\langle E_{\text{к}}\rangle = \frac32 k\cdot300{,}15 \approx 6.2 \times 10^{-21} \, \text{Дж}.
\]


\end{document}