\documentclass[12pt]{article}
\usepackage{amsmath}

\begin{document}

\section*{Сила трения}

\subsection*{Трение покоя}
Максимальная сила трения покоя пропорциональна силе нормальной реакции опоры:
\[
F_{\mathrm{тр},\max} = \mu_s N
\]
где $\mu_s$ — коэффициент трения покоя. Тело остаётся в покое при:
\[
F_{\mathrm{внеш}} \leq \mu_s N
\]

\subsection*{Трение скольжения}
При движении сила трения постоянна и направлена против скорости:
\[
F_{\mathrm{тр}} = \mu_k N
\]
где $\mu_k$ — коэффициент трения скольжения. Всегда $\mu_s > \mu_k$ из-за явления застоя.

\end{document}