\documentclass{article}
\usepackage{amsmath}

\begin{document}

\section*{Равновесие и устойчивость}

\subsection*{Основные критерии}
Точка равновесия $x_0$ определяется условием:
\[F(x_0)=0 \quad \text{или} \quad \left.\frac{dU}{dx}\right|_{x_0}=0\]

Энергетический критерий устойчивости: равновесие устойчиво при минимуме потенциальной энергии:
\[
\text{устойчиво}\ \Leftrightarrow\ U''(x_0)>0; \quad
\text{неустойчиво}\ \Leftrightarrow\ U''(x_0)<0
\]

Силовой критерий: при линеаризации силы $F(x_0+\xi)\approx \xi F'(x_0)$:
\[
\text{устойчиво}\ \Leftrightarrow\ F'(x_0)<0; \quad
\text{неустойчиво}\ \Leftrightarrow\ F'(x_0)>0
\]

\subsection*{Ключевые примеры}
Пружинный осциллятор: $U(x)=\frac12 kx^2$, устойчив при $x_0=0$

Математический маятник: $U(\theta)=mgl(1-\cos\theta)$
Устойчив при $\theta=0$ (нижнее положение), неустойчив при $\theta=\pi$ (верхнее положение)

\end{document}