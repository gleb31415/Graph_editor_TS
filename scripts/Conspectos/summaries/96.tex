\documentclass{article}
\usepackage{amsmath}

\begin{document}

\section*{Резисторы - ключевые формулы}

\subsection*{Общее сопротивление}
Эквивалентное сопротивление цепи:
\[R_0 = \frac{U_{\rm общ}}{I_{\rm общ}}\]

\subsection*{Последовательное соединение}
При последовательном соединении общее сопротивление равно сумме сопротивлений:
\[R_0 = R_1 + R_2 + \dots + R_n\]

\subsection*{Параллельное соединение}
При параллельном соединении складываются обратные величины сопротивлений:
\[\frac{1}{R_0} = \frac{1}{R_1} + \frac{1}{R_2} + \dots + \frac{1}{R_n}\]

Для двух параллельных резисторов:
\[R_0 = \frac{R_1 R_2}{R_1 + R_2}\]

\subsection*{Мощность}
Общая мощность в цепи:
\[P_0 = U_{\rm общ}I_{\rm общ} = \frac{U_{\rm общ}^2}{R_0} = I_{\rm общ}^2R_0\]

\end{document}