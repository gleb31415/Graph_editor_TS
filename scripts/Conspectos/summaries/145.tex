\documentclass{article}
\usepackage{amsmath}

\begin{document}

\section*{Термодинамика - основные формулы}

\subsection*{Нагревание и охлаждение}
Количество теплоты при изменении температуры тела массой $m$:
\[Q = cm(T_2 - T_1)\]
где $c$ - удельная теплоёмкость, $T_2-T_1$ - изменение температуры.

\subsection*{Фазовые переходы}
Теплота плавления (переход твёрдое→жидкое):
\[Q = \lambda m\]
где $\lambda$ - удельная теплота плавления.

Теплота парообразования (жидкость→пар):
\[Q = Lm\]
где $L$ - удельная теплота кипения.

Теплота сгорания топлива:
\[Q = qm\]
где $q$ - удельная теплота сгорания.

При комбинированном процессе (нагрев + плавление):
\[Q = cm(T_0 - T_1) + \lambda m\]
где $T_0$ - температура плавления.

\end{document}