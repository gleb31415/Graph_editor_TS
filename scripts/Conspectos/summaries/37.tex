\documentclass{article}
\usepackage{amsmath}

\begin{document}

\section*{Движение по криволинейной траектории}

Ускорение раскладывается на тангенциальную и нормальную составляющие:
\[\vec a = \vec a_\tau + \vec a_n\]

Нормальное ускорение при движении по окружности радиуса $r$:
\[a_n = \frac{v^2}{r} = \omega v\]

Угловое ускорение и его связь с угловой скоростью при постоянном $\alpha$:
\[\alpha = \frac{d\omega}{dt}, \quad \omega(t) = \omega_0 + \alpha t\]

Угол поворота при постоянном угловом ускорении:
\[\Delta \varphi(t) = \omega_0t + \tfrac12\alpha t^2\]

Для произвольной криволинейной траектории:
\[\vec a = \dot v\,\vec e_\tau + \frac{v^2}{R}\vec e_n\]
где $R$ — радиус кривизны, $\vec e_\tau$ и $\vec e_n$ — единичные векторы по касательной и нормали.

\end{document}