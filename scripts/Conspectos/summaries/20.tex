\documentclass[12pt]{article}
\usepackage{amsmath}

\begin{document}

\section*{Механическая энергия}

\subsection*{Кинетическая энергия}
Энергия движения тела массы $m$ со скоростью $v$:
\[E_k = \frac{m v^2}{2}\]

\subsection*{Работа силы}
Работа силы $F$ при перемещении на расстояние $x$ под углом $\theta$:
\[A = Fx\cos\theta\]

\subsection*{Теорема о кинетической энергии}
Работа равна изменению кинетической энергии:
\[A = E_{k2} - E_{k1} = \frac{m v_2^2}{2} - \frac{m v_1^2}{2}\]

\subsection*{Потенциальная энергия}
Связь силы с потенциальной энергией:
\[F = -\frac{\Delta U}{\Delta x}\]

Основные формулы потенциальной энергии:
\begin{itemize}
\item Упругой пружины: \[U = \frac{k x^2}{2}\]
\item В поле тяжести: \[U = mgh\]
\item Гравитационная: \[U = -G\frac{M m}{r}\]
\end{itemize}

\subsection*{Закон сохранения энергии}
Для потенциальных сил без потерь:
\[E_k + U = \text{const}\]

\end{document}