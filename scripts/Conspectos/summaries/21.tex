\documentclass[12pt]{article}
\usepackage{amsmath}

\begin{document}

\section*{Связи в твёрдом теле}

\subsection*{Проекция на связь}
Для двух точек твёрдого тела, проекции их скоростей на связь равны:
\[(\vec v_2 - \vec v_1)\cdot\vec e_r = 0\]
\[v_1\cos\alpha = v_2\cos\beta\]

\subsection*{Тангенциальные компоненты}
Скорость точки твёрдого тела складывается из поступательной и вращательной:
\[\vec v_i = \vec V + \vec\omega \times \vec r_i\]

Разность тангенциальных скоростей пропорциональна расстоянию между точками:
\[v_2\sin\beta - v_1\sin\alpha = |\vec\omega|\,|\vec r_2 - \vec r_1|\]

\subsection*{Ускорения}
В начальный момент при нулевых скоростях для ускорений справедливы аналогичные соотношения:
\[a_2\cos\beta - a_1\cos\alpha = 0\]
\[a_2\sin\beta - a_1\sin\alpha \sim r\]

\end{document}