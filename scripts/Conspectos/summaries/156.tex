\documentclass{article}
\usepackage{amsmath}

\begin{document}

\section*{Термодинамические циклы}

Цикл - процесс возврата рабочего тела в начальное состояние с совершением работы. Основные соотношения:

\[ A = Q_+ - Q_- \]
\[ \eta = \frac{A}{Q_+} = 1 - \frac{Q_-}{Q_+} \]

где $A$ - работа за цикл, $Q_+$ - полученное тепло, $Q_-$ - отданное тепло, $\eta$ - КПД цикла.

Для идеального газа:
\[ C_V = \frac{i}{2}R, \quad C_P = C_V + R = \frac{i+2}{2}R \]

где $i$ - число степеней свободы, $C_V$ и $C_P$ - молярные теплоёмкости при постоянном объеме и давлении.

Для прямоугольного цикла в $pV$-координатах работа равна площади:
\[ A = \Delta p \cdot \Delta V \]

\end{document}