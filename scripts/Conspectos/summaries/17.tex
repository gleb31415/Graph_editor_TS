\documentclass{article}
\usepackage{amsmath}

\begin{document}

\section*{Давление: основные формулы}

\subsection*{Давление}
Давление - отношение перпендикулярной силы к площади поверхности:
\[p = \frac{F_\perp}{S}\]

\subsection*{Гидростатика}
Закон Паскаля: изменение внешнего давления передаётся без изменений во все точки жидкости:
\[\frac{F_1}{S_1} = \Delta p = \frac{F_2}{S_2}\]

Гидростатическое давление на глубине h:
\[p = \rho g h\]

Полное давление в открытом сосуде с учётом атмосферного:
\[p(h) = p_0 + \rho gh\]

\subsection*{Сообщающиеся сосуды}
Для разных жидкостей в сообщающихся сосудах высоты столбов обратно пропорциональны плотностям:
\[h_2 = \frac{\rho_1}{\rho_0}h_1\]

\end{document}