\documentclass{article}
\usepackage{amsmath}

\begin{document}

\section*{Законы Кирхгофа}

\subsection*{Первый закон (закон узлов)}
Алгебраическая сумма токов в узле равна нулю (входящие токи положительны, выходящие отрицательны):
\[
\sum_{k} I_k = 0
\]

\subsection*{Второй закон (закон контуров)}
Алгебраическая сумма напряжений в замкнутом контуре равна нулю:
\[
\sum_{k} U_k = 0
\]
При обходе контура: для резистора берём $-IR$, для источника напряжения $+U$ если идём от минуса к плюсу, $-U$ если наоборот.

\subsection*{Применение}
1. Выбрать направления токов
2. Для узлов: $\sum I_{\rm входящие} - \sum I_{\rm выходящие}=0$
3. Для контуров: $\sum(\pm U_k) - \sum(I_m R_m)=0$
4. Решить полученную систему уравнений

\end{document}