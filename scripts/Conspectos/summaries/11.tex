\documentclass{article}
\usepackage{amsmath}

\begin{document}

\section*{Основные законы сил}

\subsection*{Третий закон Ньютона}
Силы взаимодействия двух тел равны по модулю и противоположны по направлению:
\[\vec F_{A\to B} = -\,\vec F_{B\to A}\]

\subsection*{Равнодействующая сила}
Сумма всех сил, действующих на тело:
\[\vec F_{\rm равн} = \sum_i \vec F_i\]
Для параллельных сил в одном направлении: $F_{\rm равн} = F_1 + F_2$
В противоположных направлениях: $F_{\rm равн} = F_1 - F_2$

\subsection*{Условие равновесия}
Тело находится в покое или равномерном прямолинейном движении, когда:
\[\sum_i \vec F_i = 0\]

\end{document}