\documentclass{article}
\usepackage{amsmath}

\begin{document}

\section*{Столкновения: краткий конспект}

\subsection*{Абсолютно упругий удар}
При упругом ударе сохраняются импульс и кинетическая энергия:
\[
m_1\vec v_{1}+m_2\vec v_{2}=m_1\vec v_{1}'+m_2\vec v_{2}'
\]
\[
\frac12m_1v_{1}^2+\frac12m_2v_{2}^2=\frac12m_1{v_{1}'}^2+\frac12m_2{v_{2}'}^2
\]

Для прямого удара равных масс:
\[
v_{1}'=v_{2},\quad v_{2}'=v_{1}
\]

\subsection*{Абсолютно неупругий удар}
Тела "склеиваются", движутся с общей скоростью. Сохраняется только импульс:
\[
\vec v_c=\frac{m_1\vec v_{1}+m_2\vec v_{2}}{m_1+m_2}
\]

\subsection*{Система центра масс}
В системе отсчёта центра масс суммарный импульс равен нулю:
\[
\vec p^{(C)} = \sum_im_i(\vec v_i - \vec v_C) = 0
\]

\end{document}