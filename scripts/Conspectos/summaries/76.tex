\documentclass{article}
\usepackage{amsmath}

\begin{document}

\section*{Оптика: принцип Ферма и преломление}

\subsection*{Принцип Ферма}
Оптическая длина пути должна быть экстремальной:
\[
\mathcal S=\int n(\vec r)\,ds \quad \text{и} \quad \delta\mathcal S=0
\]
где $n=c/v$ — показатель преломления среды.

\subsection*{Закон Снеллиуса}
При переходе луча между средами с показателями преломления $n_1$ и $n_2$:
\[
n_1\sin\theta_1 = n_2\sin\theta_2
\]
где $\theta_1,\theta_2$ — углы падения и преломления относительно нормали.

\subsection*{Основные показатели преломления}
Типичные значения для желтой линии натрия:
\[
n_{\text{воздух}}\approx1.0003,\quad
n_{\text{вода}}\approx1.33,\quad
n_{\text{стекло}}\approx1.50\text{–}1.52
\]

\end{document}