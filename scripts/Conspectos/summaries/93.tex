\documentclass{article}
\usepackage{amsmath}

\begin{document}

\subsubsection*{Электрический ток и напряжение}
Сила тока - это заряд, проходящий через сечение проводника за время:
\[I = \frac{Q}{t}\]

Напряжение - работа поля по перемещению заряда между точками:
\[U = \frac{A}{q} = \varphi_1 - \varphi_2\]

\subsubsection*{Закон Ома и сопротивление}
Закон Ома связывает ток, напряжение и сопротивление:
\[U = IR \quad \text{или} \quad R = \frac{U}{I}\]

Сопротивление проводника зависит от его геометрии и материала:
\[R = \rho\frac{l}{A}\]
где $\rho$ - удельное сопротивление материала, $l$ - длина, $A$ - площадь сечения.

\end{document}