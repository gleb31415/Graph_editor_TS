\documentclass{article}
\usepackage{amsmath}

\begin{document}

\section*{Векторный метод в баллистике}

Основные уравнения движения в векторной форме:
\[
\vec v_0 + \vec gt = \vec v
\]
Связывает начальную скорость, ускорение свободного падения и конечную скорость.

\[
\vec v_0t + \frac{\vec g\,t^2}{2} = \vec s
\]
Описывает перемещение тела через начальную скорость и ускорение.

\section*{Критерий максимальной дальности}

Для максимальной дальности полета между двумя точками на разных уровнях необходимо:
\[
\vec v_0 \perp \vec v
\]
Начальная и конечная скорости должны быть перпендикулярны. Это следует из максимизации площади векторного треугольника:
\[
S_\Delta = \frac{gL}{2} = \frac{vv_0\sin\!\,\bigl(\angle(\vec v,\vec v_0)\bigr)}{2}
\]

\end{document}