\documentclass{article}
\usepackage{amsmath}
\usepackage[russian]{babel}
\usepackage[utf8]{inputenc}
\usepackage[T2A]{fontenc}

\begin{document}

\section*{Электростатика: ключевые формулы}

\subsection*{Телесный угол}
Для поверхности с плотностью заряда $\sigma$, проекция поля на выбранное направление:
\[
E_\perp = k\sigma\Omega, \quad k=\frac{1}{4\pi\varepsilon_0}
\]
где $\Omega$ - телесный угол, под которым видна поверхность из точки наблюдения.

\subsection*{Поток электрического поля}
Поток через поверхность определяется как:
\[
\Phi = \iint \vec E\cdot d\vec S = \iint E\,dS\,\cos\theta
\]
Для сферы радиуса $r$ с зарядом $q$ в центре:
\[
\Phi_{\text{сфера}}=\frac{q}{\varepsilon_0}
\]

\subsection*{Теорема Гаусса}
Поток через любую замкнутую поверхность равен сумме внутренних зарядов, делённой на $\varepsilon_0$:
\[
\boxed{\iint\limits_{\text{замк.}} \vec E\cdot d\vec S=\frac{1}{\varepsilon_0}\sum q_{\text{in}}}
\]
Внешние заряды не дают вклада в поток через замкнутую поверхность.

\end{document}