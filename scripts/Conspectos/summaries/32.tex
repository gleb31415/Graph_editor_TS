\documentclass{article}
\usepackage{amsmath}

\begin{document}

\section*{Радиан и равномерное движение по окружности}

\subsection*{Радиан}
Угол в радианах - отношение длины дуги к радиусу:
\[\varphi = \frac{s}{r}\]

Для малых углов в радианах:
\[\sin x \approx x, \quad \tan x \approx x, \quad \cos x \approx 1 - \frac{x^2}{2}\]

\subsection*{Равномерное движение по окружности}
Угловая скорость - угол поворота за единицу времени:
\[\omega = \frac{\Delta\varphi}{\Delta t}\]

Период и частота обращения:
\[T = \frac{2\pi}{\omega}, \quad f = \frac{1}{T} = \frac{\omega}{2\pi}\]

Связь линейной и угловой скорости:
\[v = \omega r\]

\subsection*{Вращающаяся система отсчёта}
Правило сложения скоростей при переходе во вращающуюся систему:
\[\vec v_O = \vec v_{O'} + \vec\omega \times \vec r\]
\[\vec v_{O'} = \vec v_O - \vec\omega \times \vec r\]

\end{document}