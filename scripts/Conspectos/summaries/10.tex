\documentclass[12pt]{article}
\usepackage{amsmath}

\begin{document}

\section*{Пружины: основные формулы}

\subsection*{Деформация и закон Гука}
Абсолютная деформация пружины (разница между нагруженной и ненагруженной длиной):
\[
\Delta x = L - L_0
\]

Закон Гука (сила упругости пропорциональна деформации):
\[
F = -k\Delta x
\]

\subsection*{Системы пружин}
Параллельное соединение (пружины рядом). Эквивалентная жёсткость:
\[
k_{\rm eq}^{\rm (пар)} = k_1 + k_2
\]

Последовательное соединение (пружины цепочкой). Эквивалентная жёсткость:
\[
k_{\rm eq}^{\rm (посл)} = \frac{k_1k_2}{k_1 + k_2}
\]
или
\[
\frac{1}{k_{\rm eq}^{\rm (посл)}} = \frac{1}{k_1} + \frac{1}{k_2}
\]

\end{document}