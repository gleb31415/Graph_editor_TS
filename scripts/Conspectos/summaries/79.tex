\documentclass{article}
\usepackage{amsmath}

\begin{document}

\section*{Тонкие линзы - краткий конспект}

\subsection*{Основные типы}
Собирающая линза фокусирует параллельные лучи в фокусе. Рассеивающая линза создает расходящийся пучок, продолжения лучей пересекаются в фокусе.

\subsection*{Типы изображений}
Действительное изображение формируется реальными лучами за линзой. Мнимое изображение формируется продолжениями лучей перед линзой.

\subsection*{Формула тонкой линзы}
Для линзы с показателем преломления $n$ и радиусами кривизны $r_1, r_2$ (внутрь линзы):
\[
\frac{1}{F}=(n-1)\left(\frac{1}{r_1}+\frac{1}{r_2}\right)
\]

В конвенции направленных радиусов ($R>0$ по ходу света):
\[
\frac{1}{F}=(n-1)\left(\frac{1}{R_1}-\frac{1}{R_2}\right)
\]

Фокусное расстояние $F>0$ для собирающей линзы, $F<0$ для рассеивающей.

\end{document}