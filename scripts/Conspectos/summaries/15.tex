\documentclass{article}
\usepackage{amsmath}

\begin{document}

\section*{Работа и мощность}

\subsection*{Работа}
Работа постоянной силы вдоль направления движения:
\[A = Fl\]
Элементарная работа и полная работа через суммирование:
\[dA = F ds, \quad A = \sum F\Delta s\]

\subsection*{Мощность}
Мощность как работа в единицу времени:
\[P = \frac{A}{\Delta t} = Fv\]

\subsection*{КПД}
Коэффициент полезного действия определяется отношением полезной работы к затраченной:
\[\eta = \frac{A_{\rm полез}}{A_{\rm затр}}\times100\%\]

Для идеального рычага с плечами $a$ и $b$:
\[F_1a = F_2b, \quad \frac{s_1}{s_2} = \frac{b}{a}\]
где $F_1, F_2$ - силы, $s_1, s_2$ - перемещения на концах рычага.

\end{document}