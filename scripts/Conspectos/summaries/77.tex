\documentclass{article}
\usepackage{amsmath}

\begin{document}

\section*{Оптические формулы}

\subsection*{Плоско-параллельная пластина}
Для луча, проходящего через пластину толщины $t$ с показателем преломления $n$:
\[\sin \theta_i = n \sin \theta_r\]
Поперечное смещение луча:
\[d = t \frac{\sin(\theta_i - \theta_r)}{\cos \theta_r}\]

\subsection*{Оптический клин}
Для клина с углом $\theta$ между гранями, основные соотношения:
\[\sin \alpha = n \sin \gamma_1, \quad n \sin \gamma_2 = \sin \beta, \quad \gamma_1 + \gamma_2 = \theta\]
Угол поворота луча:
\[\delta = \alpha + \beta - \theta\]

\subsection*{Приближение малых углов}
Для малых углов ($\theta \ll 1$) угол поворота луча в клине:
\[\delta \approx (n - 1)\theta\]
Поворот пропорционален углу клина и разности показателей преломления.

\end{document}