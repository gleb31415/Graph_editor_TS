\documentclass{article}
\usepackage{amsmath}

\begin{document}

\section*{Блоки и нити - основные формулы}

\subsection*{Неподвижный блок}
Для нерастяжимой нити через неподвижный блок, скорости концов равны по модулю и противоположны по направлению:
\[v_1 + v_2 = 0\]

\subsection*{Подвижный блок}
Для нити через подвижный блок, скорость блока $v$ связана со скоростями концов нити $v_1$ и $v_2$ соотношением:
\[v = \frac{v_1 + v_2}{2}\]

Важные частные случаи:
\begin{itemize}
\item Если один конец неподвижен ($v_2=0$), а другой движется со скоростью $u$, то блок движется со скоростью $\frac{u}{2}$
\item При движении блока со скоростью $u$ и одного конца со скоростью $3u$, второй конец движется со скоростью $-u$
\end{itemize}

\end{document}