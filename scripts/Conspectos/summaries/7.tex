\documentclass[12pt]{article}
\usepackage{amsmath}

\begin{document}

\section*{Свет и оптические явления}

\subsection*{Показатель преломления}
В среде со скоростью света $v$ показатель преломления определяется как:
\[
n = \frac{c}{v} \ge 1
\]
где $c$ - скорость света в вакууме.

\subsection*{Тень и полутень}
\begin{itemize}
\item Тень - область, куда не попадает ни один луч от источника света
\item Полутень - область, куда попадает только часть лучей от протяжённого источника
\item От точечного источника формируется только тень, от протяжённого - и тень, и полутень
\end{itemize}

\end{document}