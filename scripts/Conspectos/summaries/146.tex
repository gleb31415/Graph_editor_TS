\documentclass{article}
\usepackage{amsmath}
\usepackage{gensymb}

\begin{document}

\section*{Влажность и испарение}

\subsection*{Абсолютная влажность}
Масса водяного пара в единице объёма воздуха:
\[
\rho = \frac{m_v}{V}
\]

\subsection*{Относительная влажность}
Отношение давления пара к давлению насыщенного пара при данной температуре:
\[
\varphi = \frac{p_v}{p_{\rm sat}(T)} \times 100\% = \frac{\rho_v}{\rho_{\rm sat}(T)} \times 100\%
\]

\subsection*{Кипение и испарение}
Условие кипения - равенство давления насыщенного пара атмосферному:
\[
p_{\rm sat}(T_{\rm кип}) = p_{\rm атм}
\]

Скорость испарения пропорциональна площади и разности давлений:
\[
dm \sim S\,p_{\rm sat}(T)\,(1-\varphi)\,dt
\]

Теплота испарения:
\[
Q = Lm
\]
где $L$ - удельная теплота парообразования.

\end{document}