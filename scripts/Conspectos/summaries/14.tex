\documentclass{article}
\usepackage{amsmath}

\begin{document}

\subsubsection*{Момент силы}
Момент силы $M_O$ относительно точки $O$ равен произведению силы на её плечо $r$:
\[
M_O = Fr
\]

\subsubsection*{Условие равновесия}
Для полного равновесия твёрдого тела необходимо:
\[
\sum_i F_i = 0 \quad\text{и}\quad \sum_i M_{O,i} = \sum_i F_i\,r_i = 0
\]

Для рычага с двумя силами условие равновесия:
\[
F_1\,a = F_2\,b
\]
где $a$ и $b$ — расстояния от точки опоры.

\subsubsection*{Теорема о трёх силах}
Если на тело действуют три непараллельные силы, для равновесия их линии действия должны пересекаться в одной точке.

\end{document}