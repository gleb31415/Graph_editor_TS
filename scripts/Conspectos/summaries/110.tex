\documentclass[12pt]{article}
\usepackage{amsmath}

\begin{document}

\section*{Электростатика проводников - краткие формулы}

\subsection*{Теорема о единственности}
Распределение зарядов единственно при заданном потенциале или напряжённости. Для разностного поля по теореме Гаусса:
\[\iint \delta\vec E\cdot d\vec S=\frac{q_\delta}{\varepsilon_0}\]

\subsection*{Проводники}
В объёме проводника $\vec E = 0$. Весь избыточный заряд располагается на поверхности. Поле у поверхности направлено по нормали.

\subsection*{Проводник с полостью}
Для внутренней полости с зарядами суммарный заряд на внутренней поверхности равен:
\[q_{\text{внутр. об.}} = - \sum q_{\text{in}}\]
Внешние заряды влияют только на распределение заряда внешней оболочки, внутренние - только на внутреннюю оболочку.

\end{document}