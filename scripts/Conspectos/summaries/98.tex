\documentclass{article}
\usepackage{amsmath}

\begin{document}

\section*{Симметрия в электрических цепях}

\subsection*{«Хорошая» симметрия}
При симметрии через выводы цепи токи в симметричных ветвях равны и одинаково направлены:
\[I_1 = I_2 = I\]

Для цепи с одинаковыми резисторами R:
\[R_0 = \frac{11}{10}R\]

\subsection*{«Плохая» симметрия}
При симметрии не через выводы токи в симметричных ветвях равны по модулю, но противоположны по направлению:
\[I_1 = -I_2 = I\]

В точках на оси симметрии потенциалы равны. Для цепи с одинаковыми резисторами R:
\[R_0 = \frac{6}{5}R\]

\subsection*{Поворотная симметрия}
При повороте цепи (кроме 180°) токи в переходящих друг в друга ветвях равны. Применяется для симметричных трехмерных структур.

\end{document}