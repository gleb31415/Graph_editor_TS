\documentclass{article}
\usepackage{amsmath}

\begin{document}

\section*{Клапейрон-Клаузиус и фазовые переходы}

\subsection*{Уравнение Клапейрона-Клаузиуса}
Описывает кривую насыщения (границу между жидкой и газовой фазой):
\[
\frac{dp}{dT}=\frac{L(T)}{T\left(\frac{1}{\rho_2}-\frac{1}{\rho_1}\right)}
\]

\subsection*{Зависимость теплоты парообразования}
Изменение теплоты парообразования при постоянных теплоёмкостях:
\[
L_2 = L_1 + (4R - C)(T_2-T_1)
\]

\subsection*{Давление насыщенного пара}
При постоянной теплоте парообразования:
\[
p(T)=p_1\exp\!\left[-\frac{L}{R}\left(\frac{1}{T}-\frac{1}{T_1}\right)\right]
\]

С учётом линейной зависимости $L(T)$:
\[
p(T)=p_1\exp\!\left[\frac{L_1 - k T_1}{R}\left(\frac{1}{T_1}-\frac{1}{T}\right)\right]
\left(\frac{T}{T_1}\right)^{k/R}
\]
где $k=4R-C$.

\end{document}