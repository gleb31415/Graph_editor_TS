\documentclass[12pt]{article}
\usepackage{amsmath}

\begin{document}

\section*{Метод наложения цепей}

\subsection*{Основной принцип}
В электрической цепи между выводами A и B можно ввести вспомогательный вывод C и разложить цепь на сумму двух цепей (AC и BC). При наложении необходимо уравнять входящие и выходящие токи в узле C.

\subsection*{Расчет сопротивления}
Для эквивалентного сопротивления после наложения цепей:
\[
R_0 = \frac{U_\text{общ}}{I_\text{общ}} = \frac{17I\cdot R}{24I} = \frac{17}{24}R
\]

\subsection*{Бесконечные сетки}
В бесконечных квадратных сетках из одинаковых резисторов R, при введении вывода C "на бесконечности", эквивалентное сопротивление:
\[
R_0 = \frac{U_\text{общ}}{I_\text{общ}} = \frac{2IR}{4I} = \frac{R}{2}
\]

\end{document}