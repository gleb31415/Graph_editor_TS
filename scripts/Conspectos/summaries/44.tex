\documentclass{article}
\usepackage{amsmath}

\begin{document}

\section*{Вязкое трение}

\subsection*{Основные режимы}
Сила вязкого сопротивления в двух режимах:
\[
\begin{cases}
\vec F_{\rm вяз} = -k_1\vec v, & \text{ламинарный режим}\\
\vec F_{\rm вяз} = -k_2v\vec v, & \text{турбулентный режим}
\end{cases}
\]

Число Рейнольдса определяет режим течения:
\[ \mathrm{Re} = \frac{\rho v L}{\eta} \]

Закон Стокса для малого шара в ламинарном режиме:
\[ \vec F_{\rm Стокса} = -6\pi \eta R\,\vec v \]

\subsection*{Движение с линейным сопротивлением}
Скорость и координата со временем:
\[ v(t) = v_0e^{-\tfrac{k_1}{m}t} \]
\[ x(t) = \frac{m}{k_1}\,v_0(1 - e^{-\tfrac{k_1}{m}t}) \]

\subsection*{Движение с квадратичным сопротивлением}
Скорость и координата со временем:
\[ v(t) = \frac{v_0}{1 + \tfrac{k_2}{m}v_0\,t} \]
\[ x(t) = \frac{m}{k_2}\,\ln\,\!(1 + \tfrac{k_2}{m}v_0\,t) \]

\end{document}