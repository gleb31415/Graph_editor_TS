\documentclass{article}
\usepackage{amsmath}

\begin{document}

\section*{Телесные углы - основные формулы}

\subsection*{Определение}
Телесный угол $\Omega$ - отношение площади участка на сфере к квадрату радиуса:
\[\Omega=\frac{S}{r^2}\]

Полная сфера имеет телесный угол $4\pi$, полусфера - $2\pi$.

\subsection*{Элемент телесного угла}
В сферических координатах элемент телесного угла:
\[d\Omega=\sin\theta\,d\theta\,d\varphi\]

\subsection*{Конус}
Телесный угол конуса с полууглом $\alpha$:
\[\Omega_{\text{конус}}=2\pi\,(1-\cos\alpha)\]

Для малых углов ($\alpha \ll 1$):
\[\Omega\approx \pi\alpha^2\]

\subsection*{Практические формулы}
Для удаленного малого объекта площадью $A$ под углом $\beta$:
\[\Omega\approx \frac{A\cos\beta}{R^2}\]

Доля полной сферы:
\[f=\frac{\Omega}{4\pi}\]

\end{document}