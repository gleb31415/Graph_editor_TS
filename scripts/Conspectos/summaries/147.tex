\documentclass{article}
\usepackage{amsmath}

\begin{document}

\section*{Тепловой баланс}

Основной закон теплового баланса: сумма всех теплот в замкнутой системе равна нулю
\[
\sum_i Q_i = 0
\]

Формулы для количества теплоты при нагреве и фазовых переходах:
\[
Q = cm(T_{\rm конечная}-T_{\rm начальная}), \quad Q = \lambda m
\]
где $c$ - удельная теплоёмкость, $m$ - масса, $\lambda$ - удельная теплота фазового перехода.

Общее уравнение теплового баланса для системы с нагревом и фазовыми переходами:
\[
\sum_{\rm нагрев} c_i m_i(T_k - T_i) + \sum_{\rm плавление/кипение} \lambda_j m_j = 0
\]
где $T_k$ - конечная равновесная температура.

\end{document}