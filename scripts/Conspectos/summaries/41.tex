\documentclass{article}
\usepackage{amsmath}

\begin{document}

\section*{Сухое трение}

Максимальная сила трения покоя пропорциональна силе нормальной реакции:
\[
F_{\mathrm{тр},\max} = \mu_s N
\]

При скольжении сила трения постоянна и направлена против движения:
\[
F_{\mathrm{тр}} = \mu_k N
\]
где $\mu_k < \mu_s$ (коэффициент трения скольжения меньше коэффициента трения покоя).

Работа силы трения на малом перемещении:
\[
dA = -F_{\mathrm{тр}}\,dx = -\mu_kN\,dx
\]

\section*{Качение без проскальзывания}

Связь между линейной и угловой скоростями при качении без проскальзывания:
\[
\omega = \frac{v_0}{R}
\]
где $v_0$ - скорость центра, $R$ - радиус, $\omega$ - угловая скорость.

\end{document}