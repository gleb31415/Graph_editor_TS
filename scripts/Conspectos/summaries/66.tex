\documentclass{article}
\usepackage{amsmath}

\begin{document}

\section*{Key Calculus Concepts}

\subsection*{Derivatives}
Derivative is the rate of change of a function:
\[f'(x)=\lim_{h\to0}\frac{f(x+h)-f(x)}{h} = \frac{df}{dx}\]

\subsection*{Basic Derivative Rules}
Sum rule: $(f+g)'=f'+g'$

Product rule: $(fg)'=f'g+fg'$

Quotient rule: $\left(\frac{f}{g}\right)'=\frac{f'g-fg'}{g^2}$

Chain rule: $(f\circ g)'(x)=f'(g(x))g'(x)$

\subsection*{Common Derivatives}
Power rule: $\frac{d}{dx}x^n=nx^{n-1}$

Trigonometric:
\[\frac{d}{dx}\sin x=\cos x \quad \frac{d}{dx}\cos x=-\sin x \quad \frac{d}{dx}\tan x=\sec^2 x\]

\subsection*{Critical Points}
For extrema, find where $f'(x)=0$. Then:
\[
\begin{cases}
f''(x_0)>0 & \text{minimum at }x_0\\
f''(x_0)<0 & \text{maximum at }x_0
\end{cases}
\]

\end{document}