\documentclass{article}
\usepackage{amsmath}

\begin{document}

\section*{Влажный воздух - краткие формулы}

\subsection*{Водяной пар}
При температуре кипения воды давление насыщенного пара:
\[p_\text{sat}(T = 100^\circ C) = p_0 \approx 100\; \text{кПа}\]

\subsection*{Основные соотношения}
Для сверхнагретого пара (до насыщения):
\[p(V)=\frac{\nu R T}{V}, \quad \text{пока} \quad p < p_{\text{sat}}(T)\]

В двухфазной области (конденсация):
\[p(V)=p_{\text{sat}}(T)=\text{const}, \quad V\in[V_{\ell},V_{\text{sat}}]\]

\subsection*{Влажный воздух}
Для смеси сухого воздуха и пара до насыщения:
\[p(V)=\frac{(\nu_d+\nu_v) R T}{V}\]

После насыщения:
\[p(V)=\frac{\nu_d R T}{V}+p_{\text{sat}}(T)\]

Количество пара в газовой фазе:
\[\nu_{\text{пар}}(V)=\frac{p_{\text{sat}}(T)}{R T}\,V\]

\end{document}