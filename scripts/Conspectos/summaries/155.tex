\documentclass{article}
\usepackage{amsmath}

\begin{document}

\section*{Политропический процесс}

Молярная теплоёмкость и первое начало термодинамики для политропического процесса:
\[
C=\frac{1}{\nu}\frac{dQ}{dT}, \quad dQ=dU+p\,dV
\]

Основное уравнение политропы с показателем $n$:
\[
p\,V^n=\text{const}, \quad \text{где} \quad n=\frac{C-C_p}{C-C_V}
\]

Связанные инварианты политропы:
\[
T\,V^{n-1}=\text{const}, \quad p^{1-n}\,T^n=\text{const}
\]

Частные случаи:
\begin{itemize}
\item Изобара: $C=C_p$, $n=0$, $p=\text{const}$
\item Изотерма: $C\to\infty$, $n=1$, $pV=\text{const}$
\item Изохора: $C=C_V$, $V=\text{const}$
\item Адиабата: $C=0$, $n=\gamma=\frac{C_p}{C_V}$
\end{itemize}

\end{document}