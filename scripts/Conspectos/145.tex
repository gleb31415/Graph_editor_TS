\documentclass[12pt, a4paper]{article}% тип документа, размер шрифта
\usepackage[T2A]{fontenc}%поддержка кириллицы в ЛаТеХ
\usepackage[utf8]{inputenc}%кодировка
\usepackage[russian]{babel}%русский язык
\usepackage{mathtext}% русский текст в формулах
\usepackage{amsmath}%удобная вёрстка многострочных формул, масштабирующийся текст в формулах, формулы в рамках и др.
\usepackage{amsfonts}%поддержка ажурного и готического шрифтов — например, для записи символа {\displaystyle \mathbb {R} } \mathbb {R} 
\usepackage{amssymb}%amsfonts + несколько сотен дополнительных математических символов
\frenchspacing%запрет длинного пробела после точки
\usepackage{setspace}%возможность установки межстрочного интервала
\usepackage{indentfirst}%пакет позволяет делать в первом абзаце после заголовка абзацный отступ
\usepackage[unicode, pdftex]{hyperref}
\onehalfspacing%установка полуторного интервала по умолчанию
\usepackage{graphicx}%подключение рисунков
\graphicspath{{images/}}%путь ко всем рисункам
\usepackage{caption}
\usepackage{float}%плавающие картинки
\usepackage{tikz} % это для чудо-миллиметровки
\usepackage{pgfplots}%для построения графиков
\pgfplotsset{compat=newest, y label style={rotate=-90},  width=10 cm}%версия пакета построения графиков, ширина графиков
\usepackage{pgfplotstable}%простое рисование табличек
\usepackage{lastpage}%пакет нумерации страниц
\usepackage{comment}%возможность вставлять большие комменты
\usepackage{float}
%%%%% ПОЛЯъ
\setlength\parindent{0pt} 
\usepackage[top = 2 cm, bottom = 2 cm, left = 1.5 cm, right = 1.5 cm]{geometry}
\setlength\parindent{0pt}
%%%%% КОЛОНТИТУЛЫ
\usepackage{xcolor}
\usepackage{amsmath}
\usepackage{gensymb}
\usepackage{tikz}

\begin{document}


\subsubsection*{Важные понятия термодинамики}
\textit{Внутренняя энергия тела} $U$ — это сумма кинетической энергии хаотического движения частиц и потенциальной энергии их взаимодействий. При нагревании или охлаждении тела его внутренняя энергия меняется.

\textit{Теплопередача} — это процесс передачи тепловой энергии от более нагретого тела (или его части) к менее нагретому.

\textit{Количество теплоты} $Q$ — это энергия, переданная телу или отданная им в процессе теплопередачи. По знаку $Q>0$, если тело получило энергию, и $Q<0$, если отдало.


\subsubsection*{Нагревание и охлаждение}
\textit{Удельная теплоёмкость} $c$ — это количество теплоты, необходимое для изменения температуры единицы массы вещества на $1^\circ\!C$ (или 1 K). При нагревании тела массой $m$ от температуры $T_1$ до $T_2$:
\[
Q = cm(T_2 - T_1),
\]
где
\begin{itemize}
  \item $m$ — масса тела (кг);
  \item $c$ — удельная теплоёмкость вещества (Дж / (кг · K));
  \item $T_2 - T_1$ — изменение температуры (K или $^\circ C$).
\end{itemize}

Иногда как характеристику конкретного тела вводят и просто теплоёмкость --- количество теплоты, необходимое для изменения температуры тела на $1^\circ\!C$ (или 1 K). Однако эта величина не используется часто, поскольку не является удобной (удельная теплоёмкость более универсальна, так как является характеристикой вещества, а не тела). Теплоёмкость однородного тела $C$ выражается через его массу и удельную теплоёмкость его вещества:

\[
C = cm.
\]

\subsubsection*{Фазовые переходы}

\textit{Фазовый переход} — превращение вещества из одного агрегатного состояния в другое (твёрдое, жидкое или газообразное).
При фазовом переходе температура остаётся постоянной (процесс идёт при фиксированной $T$), но внутренняя энергия меняется:
часть энергии идёт на разрыв или установление межмолекулярных связей. При этом изменяется потенциальная энергия взаимодействия молекул.


\textit{Плавление} — это переход вещества из твёрдого состояния в жидкое при постоянной температуре плавления. \textit{Удельная теплота плавления} $\lambda$ — это количество теплоты, необходимое для превращения единицы массы вещества из твёрдого состояния в жидкое без изменения температуры. Для плавления массы $m$:
\[
Q = \lambda m,
\]
где
\begin{itemize}
  \item $\lambda$ — удельная теплота плавления (Дж / кг).
\end{itemize}

\textit{Кипение} — это переход вещества из жидкого состояния в газообразное при постоянной температуре (температуре кипения). \textit{Удельная теплота кипения} $L$ — это количество теплоты, необходимое для превращения единицы массы вещества из жидкого состояния в пар без изменения температуры:
\[
Q = Lm,
\]
где
\begin{itemize}
  \item $L$ — удельная теплота кипения (Дж / кг).
\end{itemize}

\textit{Удельная теплота сгорания} $q$ — это количество энергии, выделяющееся при полном сгорании единицы массы или объёма топлива. Для массы топлива $m$:
\[
Q = qm,
\]
где
\begin{itemize}
  \item $q$ — удельная теплота сгорания (Дж / кг).
\end{itemize}

\textbf{Пример.} 

Тело массой $m$ в твёрдом состянии нагрели с $T_1$ до $T_0$ --- температуры плавления этого вещества, затем расплавили. Общее количество тепла, переданного телу, равно
\[
Q = cm(T_0 - T_1) + \lambda m.
\]



\end{document}