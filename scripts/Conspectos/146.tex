\documentclass[12pt, a4paper]{article}% тип документа, размер шрифта
\usepackage[T2A]{fontenc}%поддержка кириллицы в ЛаТеХ
\usepackage[utf8]{inputenc}%кодировка
\usepackage[russian]{babel}%русский язык
\usepackage{mathtext}% русский текст в формулах
\usepackage{amsmath}%удобная вёрстка многострочных формул, масштабирующийся текст в формулах, формулы в рамках и др.
\usepackage{amsfonts}%поддержка ажурного и готического шрифтов — например, для записи символа {\displaystyle \mathbb {R} } \mathbb {R} 
\usepackage{amssymb}%amsfonts + несколько сотен дополнительных математических символов
\frenchspacing%запрет длинного пробела после точки
\usepackage{setspace}%возможность установки межстрочного интервала
\usepackage{indentfirst}%пакет позволяет делать в первом абзаце после заголовка абзацный отступ
\usepackage[unicode, pdftex]{hyperref}
\onehalfspacing%установка полуторного интервала по умолчанию
\usepackage{graphicx}%подключение рисунков
\graphicspath{{images/}}%путь ко всем рисункам
\usepackage{caption}
\usepackage{float}%плавающие картинки
\usepackage{tikz} % это для чудо-миллиметровки
\usepackage{pgfplots}%для построения графиков
\pgfplotsset{compat=newest, y label style={rotate=-90},  width=10 cm}%версия пакета построения графиков, ширина графиков
\usepackage{pgfplotstable}%простое рисование табличек
\usepackage{lastpage}%пакет нумерации страниц
\usepackage{comment}%возможность вставлять большие комменты
\usepackage{float}
%%%%% ПОЛЯъ
\setlength\parindent{0pt} 
\usepackage[top = 2 cm, bottom = 2 cm, left = 1.5 cm, right = 1.5 cm]{geometry}
\setlength\parindent{0pt}
%%%%% КОЛОНТИТУЛЫ
\usepackage{xcolor}
\usepackage{amsmath}
\usepackage{gensymb}
\usepackage{tikz}

\begin{document}


\subsubsection*{Абсолютная влажность}
\textit{Абсолютная влажность} $\rho$ — это масса водяного пара $m_v$, содержащегося в единице объёма $V$ воздуха, то есть плотность пара:
\[
\rho = \frac{m_v}{V},
\]
где
\begin{itemize}
  \item $m_v$ — масса водяного пара (кг);
  \item $V$ — объём воздуха (м$^3$);
  \item $\rho$ — абсолютная влажность (кг/м$^3$).
\end{itemize}

Давление пара $p_v$ — это частичное давление водяного пара в смеси газов. 

Из термодинамики известно, что при постоянной температуре давление газа пропорционально его плотности, поэтому $p_v \sim
 \rho$.

При данной температуре существует насыщенный пар — это такое состояние, когда пар находится в равновесии с жидкой фазой.
Дальнейшее испарение без увеличения температуры невозможно (скорость испарения равна скорости возвращения молекул пара в жидкость).
В этот момент достигается максимальное \textit{давление насыщенного пара} $p_{sat}(T)$ и вместе с этим максимальная плотность
$\rho_{sat}(T)$ --- плотность насыщенного пара. Эти величины для определённой температуры в рамках восьмого класса определяются по
таблицам измеренных значений из интернета или справочников.

\subsubsection*{Относительная влажность}

\textit{Относительная влажность} $\varphi$ показывает, какую долю от максимального давления пара $p_{sat}(T)$ при данной температуре составляет реальное давление пара:
\[
\varphi = \frac{p_v}{p_{\rm sat}(T)} \times 100\textpercent{ } = \frac{\rho_v}{\rho_{\rm sat}(T)} \times 100\textpercent{ } ,
\]
где
\begin{itemize}
  \item $p_v$ — фактическое давление водяного пара (Па);
  \item $p_{\rm sat}(T)$ — давление насыщенного пара при той же температуре (Па);
  \item $\varphi$ — относительная влажность.
\end{itemize}

\subsubsection*{Критерий кипения}

При достижении температуры кипения жидкости её давление насыщенного пара становится равным внешнему (атмосферному) давлению.
До этой точки пары в жидкости, образующиеся при нагревании, не могут преодолеть давление сверху и сжимаются обратно. 
Когда же $p_{sat}(T_{\rm кип}) = p_{\rm атм}$, пузырьки пара свободно растут по всему объёму жидкости, а не только у поверхности, и начинаются активные кипение и парообразование. Иначе говоря, точка кипения определяется условием  
\[
p_{\rm sat}(T_{\rm кип}) = p_{\rm атм},
\]
потому что только при таком давлении паровые пузырьки могут преодолеть силу внешнего давления и выйти в объём окружающей среды.

\subsubsection*{Испарение}
\textit{Испарение} — самопроизвольный переход молекул с поверхности жидкости в паровую фазу. Оно происходит при любой температуре: у молекул в жидкости распределены скорости, и часть самых быстрых покидает поверхность, преодолевая притяжение соседей. Испарение — поверхностный процесс, в отличие от кипения, которое идёт по всему объёму.

Ключевые факты:
\begin{itemize}
  \item При фиксированной температуре испарение продолжается, пока пар над жидкостью \underline{ненасыщен}: \(p_v < p_{\rm sat}(T)\). В закрытом сосуде устанавливается \textit{динамическое равновесие} при \(p_v=p_{\rm sat}(T)\): скорости испарения и конденсации равны.
  \item Скорость испарения возрастает с температурой, площадью поверхности и уносом пара потоком воздуха, и убывает с ростом относительной влажности. Для малых промежутков времени удобно записывать пропорциональность
  \[
  dm \sim S\,[\,p_{\rm sat}(T)-p_v\,]\,dt \sim S\,p_{\rm sat}(T)\,(1-\varphi)\,dt,
  \]
  где \(m\) — испарившаяся масса за время \(dt\), \(S\) — площадь поверхности, \(\varphi\) — относительная влажность.
  \item На испарение требуется тепло: чтобы испарить массу \(m\) жидкости, нужно количество теплоты
  \[
  Q = Lm,
  \]
  где \(L\) — \textit{удельная теплота парообразования}. Если тепло извне не подводится, жидкость и контактирующая с ней поверхность охлаждаются (поэтому мокрая кожа зябнет на ветру).
\end{itemize}

Связь с кипением: при \(p_{\rm sat}(T_{\rm кип})=p_{\rm атм}\) паровые пузырьки могут расти и в глубине жидкости, и процесс переходит из поверхностного испарения в объёмное парообразование (кипение).


\end{document}