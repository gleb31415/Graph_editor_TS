\documentclass[12pt, a4paper]{article}% тип документа, размер шрифта
\usepackage[T2A]{fontenc}%поддержка кириллицы в ЛаТеХ
\usepackage[utf8]{inputenc}%кодировка
\usepackage[russian]{babel}%русский язык
\usepackage{mathtext}% русский текст в формулах
\usepackage{amsmath}%удобная вёрстка многострочных формул, масштабирующийся текст в формулах, формулы в рамках и др.
\usepackage{amsfonts}%поддержка ажурного и готического шрифтов — например, для записи символа {\displaystyle \mathbb {R} } \mathbb {R} 
\usepackage{amssymb}%amsfonts + несколько сотен дополнительных математических символов
\frenchspacing%запрет длинного пробела после точки
\usepackage{setspace}%возможность установки межстрочного интервала
\usepackage{indentfirst}%пакет позволяет делать в первом абзаце после заголовка абзацный отступ
\usepackage[unicode, pdftex]{hyperref}
\onehalfspacing%установка полуторного интервала по умолчанию
\usepackage{graphicx}%подключение рисунков
\graphicspath{{images/}}%путь ко всем рисункам
\usepackage{caption}
\usepackage{float}%плавающие картинки
\usepackage{tikz} % это для чудо-миллиметровки
\usepackage{pgfplots}%для построения графиков
\pgfplotsset{compat=newest, y label style={rotate=-90},  width=10 cm}%версия пакета построения графиков, ширина графиков
\usepackage{pgfplotstable}%простое рисование табличек
\usepackage{lastpage}%пакет нумерации страниц
\usepackage{comment}%возможность вставлять большие комменты
\usepackage{float}
%%%%% ПОЛЯъ
\setlength\parindent{0pt} 
\usepackage[top = 2 cm, bottom = 2 cm, left = 1.5 cm, right = 1.5 cm]{geometry}
\setlength\parindent{0pt}
%%%%% КОЛОНТИТУЛЫ
\usepackage{xcolor}
\usepackage{amsmath}
\usepackage{gensymb}
\usepackage{tikz}

\begin{document}



\subsubsection*{Вязкое трение}
\textit{Вязкое трение} возникает в результате внутреннего трения в жидкости или газе при движении тела через среду.  
Сила вязкого сопротивления зависит от скорости и режима обтекания:
\[
\begin{cases}
\vec F_{\rm вяз} = -k_1\vec v \sim v, & \text{ламинарный (стоксовский) режим},\\
\vec F_{\rm вяз} = -k_2v\vec v \sim v^2, & \text{турбулентный режим},
\end{cases}
\]
где $k_1$ и $k_2$ — коэффициенты, зависящие от вязкости среды, формы и размеров тела. Ламинарный режим характерен для небольших скоростей и малых размеров, турбулентный — при больших скоростях или размерах.



\subsubsection*{Режимы обтекания и закон Стокса}
Удобный безразмерный параметр — \textit{число Рейнольдса} \( \mathrm{Re} = \dfrac{\rho v L}{\eta} \) (плотность \(\rho\), характерный размер \(L\), динамическая вязкость вещества \(\eta\)). При \(\mathrm{Re}\ll1\) течение «вязкое и тихое» (ламинарное), при больших \(\mathrm{Re}\) — переход к турбулентности.

Для маленького гладкого шарика радиуса \(R\) в вязкой несжимаемой жидкости при \(\mathrm{Re}\ll1\) действует \textit{закон Стокса}:
\[
\vec F_{\rm Стокса} = -6\pi \eta R\,\vec v,
\]
то есть в модели \(\vec F_{\rm вяз} = -k_1\vec v\) имеем \(k_1 = 6\pi \eta R\).
Грубая граница смены режимов по скорости можно оценить из равенства вкладов \(k_1 v \sim k_2 v^2\):
\[
v_\ast \sim \frac{k_1}{k_2}.
\]
При \(v\ll v_\ast\) доминирует линейный закон, при \(v\gg v_\ast\) — квадратичный.


\subsubsection*{Движение под действием линейного сопротивления ($F=-k_1v$)}

Зависимость скорости от пройденного пути:
\[
\frac{dv}{dx} = \frac{dv/dt}{dx/dt} = -\frac{k_1v}{mv}
\quad\Longrightarrow\quad
v(x) = v_0 - \frac{k_1}{m}\,x.
\]

Обнуление скорости соответствует максимальному перемещению с начальной скоростью $v_0$:
\[
x_0(v_0) = \frac{mv_0}{k_1}.
\]


Теперь для Из второго закона Ньютона без других сил:
\[
m\,\frac{dv}{dt} = -k_1v.
\]
Разделим переменные и интегрируем:
\[
\frac{dv}{v} = -\frac{k_1}{m}\,dt
\quad\Longrightarrow\quad
v(t) = v_0e^{-\tfrac{k_1}{m}t}.
\]
Этот интеграл не обязан уметь брать школьник девятого класса, однако он полезен для понимания процесса.
Координата:
\[
dx = v\,dt
\;\Longrightarrow\;
x(t) = \int_0^t v_0\,e^{-\tfrac{k_1}{m}t'}\,dt'
= \frac{m}{k_1}\,v_0\bigl(1 - e^{-\tfrac{k_1}{m}t}\bigr).
\]




\subsubsection*{Движение под действием квадратичного сопротивления ($F=-k_2v^2$)}
\[
m\,\frac{dv}{dt} = -k_2\,v^2
\quad\Longrightarrow\quad
\frac{dv}{v^2} = -\frac{k_2}{m}\,dt
\quad\Longrightarrow\quad
v(t) = \frac{v_0}{1 + \tfrac{k_2}{m}v_0\,t}.
\]
Координата:
\[
dx = v\,dt
\;\Longrightarrow\;
x(t) = \int_0^t \frac{v_0\,dt'}{1 + \tfrac{k_2}{m}v_0\,t'}
= \frac{m}{k_2}\,\ln\,\!\bigl(1 + \tfrac{k_2}{m}v_0\,t\bigr).
\]
Зависимость скорости от пройденного пути:
\[
\frac{dv}{dx} = -\frac{k_2}{m}\,v
\quad\Longrightarrow\quad
v(x) = v_0\,e^{-\tfrac{k_2}{m}x}.
\]
Как видим, для этого режима предела по $x$ будто бы нет, однако очевидно что по мере понижения скорости режим постепенно вернётся к линейному и тело в какой то момент остановистя.

\end{document}